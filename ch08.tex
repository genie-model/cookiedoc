%----------------------------------------------------------------------------------------
%       CHAPTER 8
%----------------------------------------------------------------------------------------

\cleardoublepage

\chapterimage{spore-920.png} % Chapter heading image

\chapter{Marine ecosystems and dynamics}\label{ch:marine-ecosystems}

\hfill \break

\noindent In the chapter we address the role and nature of marine (plankton) ecosystems.\footnote{Loosely based on original workshop material  devised by Ben Ward <b.a.ward@soton.ac.uk>}

\vspace{1mm}
\textbf{muffin} includes an explicit ecosystem component, including primary and export production as well as plankton biomass -- \textbf{ECOGEM}\footnote{\href{https://doi.org/10.5194/gmd-2017-258}{\textit{Ward et al.} [2017]} -- Ward, B. A., Wilson, J. D., Death, R. M., Monteiro, F. M., Yool, A., and Ridgwell, A.: EcoGEnIE 0.1: Plankton Ecology in the cGENIE Earth system model, \textit{Geosci. Model Dev. Discuss.}, https://doi.org/10.5194/gmd-2017-258, 2017.} -- that is designed as an alternative option to the 'bioloigically induced export flux' representation of export production (\textbf{BIOGEM}). The ecological model takes what is known as a size-structured approach to representing diversity of function in marine ecosystems, and is flexible in being able to be configured to represent any range of size classes of phytoplankton and zooplankton (and/or mixotrophs).

\section*{Stuff to keep in mind\dots}
\begin{itemize}
\item We will be working with highly idealised ecosystems in a relatively idealised (modern) ocean.
\item The aim is to explore why the model behaves as it does.
\item The assumption is that this will give us some insight into why the real world behaves as it does. Perhaps. (It is up to you to question the validity of this assumption.)
\end{itemize}

%------------------------------------------------
\newpage
%------------------------------------------------

\section*{Read.me}

\vspace{2mm}
You will need a \textit{re-start} file to carry out the experiments in the Chapter:
\small\begin{verbatim}
$ wget --no-check-certificate http://www.seao2.info/cgenie_output/ ...
muffin.CBE.p_worjh2.BASESFeTDTL.FeMIP.SPIN.tar.gz
\end{verbatim}\normalsize

\noindent Extract the results in the usual way and in the usual place ... and return to \textsf{\footnotesize genie-main} in the usual way ... all ... as ... usual.

%------------------------------------------------
\newpage
%------------------------------------------------

\section{Getting going with ECOGEM}

Previously, you were running the standard 'biogeochemical' version of \textbf{muffin}\footnote{e.g. see \textit{Ridgwell et al.} [2007]}.  In \textbf{BIOGEM}, the biological pump is driven by an implicit (i.e. unresolved) biological community. As in the real ecosystem, the biological uptake of carbon and nutrients (such as phosphorus and iron) is limited by light, temperature and nutrient availability. However, unlike the real ecosystem, any uptake is \textit{directly} and \textit{instantly} converted to particulate and dissolved organic matter (POM, DOM) and exported to the ocean interior via (gravitational) settling and the ocean surface, respectively. i.e.
\vspace{4mm}
\begin{itemize}
\item \underline{surface inorganic nutrients} $\xrightarrow[\rm and~export]{\rm production}$ \underline{POM and DOM}
\end{itemize}
\vspace{4mm}

In contrast, in this chapter you we are going get started with the '\textbf{ECOGEM}' ecological modeling package\footnote{see: \href{https://doi.org/10.5194/gmd-2017-258}{\textit{Ward et al.} [2017]} -- Ward, B. A., Wilson, J. D., Death, R. M., Monteiro, F. M., Yool, A., and Ridgwell, A.: EcoGEnIE 0.1: Plankton Ecology in the cGENIE Earth system model, \textit{Geosci. Model Dev. Discuss.}, https://doi.org/10.5194/gmd-2017-258, 2017.}. This will allow us to extend the capabilities of \textbf{muffin} to examine a range of questions relating to the role of physiology and community structure in regulating the biological pump and hence atmospheric \(CO_{2}\) etc. In \textbf{ECOGEM}, biological uptake is again limited by light, temperature and nutrient availability ... but now it must pass through an explicit and dynamic intermediary plankton biomass pool before the net products of biological production can expressed as the production of POM and DOM:
\vspace{4mm}
\begin{itemize}
\item \underline{surface inorganic nutrients} $\xrightarrow[]{\rm production}$ \underline{plankton biomass} $\xrightarrow[]{\rm export}$ \underline{POM and DOM}
\end{itemize}
\vspace{2mm}

\vspace{1mm}
The existence of a plankton biomass reservoir creates a delay term in the system such that the occurrence of warm temperatures in a sunlit surface with abundant nutrients, does not in itself guarantee immediate and massive carbon export. This is because the ecosystem biomass must build up first. This could be important for e.g. the timing of spring blooms.

\vspace{1mm}
Note that while the experimental configurations are based on those of \textit{Ward et al.} [2018], here we use a slightly different modern continental configuration and physics tuning (and hence ocean circulation state) and a slightly different iron cycle tuning. \textbf{ECOGEM} itself has also been adjusted to reduce the carbon export relative to phosphorous and has photosynthesis suppressed under sea-ice.

%------------------------------------------------
\vspace{1mm}
\noindent\rule{4cm}{0.1mm}
%------------------------------------------------

\subsubsection{Running the model}
\vspace{1mm}

We will start with the simplest possible configuration of \textbf{ECOGEM}, with just a single (small) phytoplankton class\footnote{i.e. as if the ocean was populated with a small species of phytoplankton (photo-autotroph) and nothing else.}. Run this model at the command line (e.g. for 10 years), as follows:
\vspace{-1mm}\small\begin{verbatim}
$ ./runmuffin.sh muffin.CBE.p_worjh2.BASESFeTDTL LABS EXP.8.1 10 
   muffin.CBE.p_worjh2.BASESFeTDTL.FeMIP.SPIN
\end{verbatim}\normalsize\vspace{-1mm}
(Here, you are using a new \textit{base-config} -- \textsf{\footnotesize muffin.CBE.p\_worjh2.BASESFeTDTL} -- identical to the iron-enabled configuration used in the previous Chapter, but now with the ecosystem model enabled.)

The model will run as before ... except much slower ... :( ... as \textbf{muffin} has to calculate plankton growth and ecological interactions in addition to everything else it did previously.

%------------------------------------------------
\newpage
%------------------------------------------------
%
\subsubsection{Viewing 2D time-slice output}
\vspace{1mm}

What is 'new'?

\vspace{1mm}
\noindent Following the same convention as for \textbf{BIOGEM}, \textbf{ECOGEM} \textit{time-slice} output is saved in the subdirectory of your experiment results directory, named \footnotesize\textsf{ecogem }\normalsize.\footnote{for this particular example experiment, the full path to the \textbf{ECOGEM} results would be: \\ \scriptsize\textsf{/cgenie\_output/EXP.8.1/ecogem/}\normalsize.}

\begin{enumerate}[noitemsep]
\vspace{1mm}
\item Open the \textsf{\footnotesize fields\_ecogem\_2D.nc} file by locating it in the correct directory, and double clicking on it in the file transfer window (if you have that software function configured), or transfer locally and then open (e.g. in \textbf{Panoply}).
\vspace{1mm}
\item You should now see a list of 2D arrays that were output by \textbf{ECOGEM}. Looking at the \textsf{\footnotesize Long Name} description, simply click on a variable of interest. If a menu window pops up, just click on \textsf{\footnotesize Create} or hit the \textsf{\footnotesize Return} key.
\vspace{1mm}
\item Check the \textbf{Panoply} settings to make sure you really know what you are looking at.
\vspace{1mm}
\begin{itemize}[noitemsep]
\item Which \textit{time-slice} (i.e. simulation year) are you looking at?
\item What is the data range (i.e. colour scale)
\end{itemize}
\end{enumerate}
\vspace{2mm}

\noindent A guide to some key \textbf{ECOGEM} output variables  can be found at the end of Chapter \ref{ch:model-output}.

%------------------------------------------------
\vspace{1mm}
\noindent\rule{4cm}{0.1mm}
%------------------------------------------------

\subsubsection{Comparing to observations}
\vspace{1mm}

Models are intended as an (as close as possible) approximation of the real world (whatever that is). It might, therefore, be useful to check if our approximation is in anyway realistic. We can do this by comparing the model output\footnote{NOTE: \textbf{ECOGEM} only saves a limited number of surface (2D) data arrays. You can look at other variables (in 2D and 3D) by opening the corresponding \textbf{BIOGEM} netCDF files.} to observations.

\vspace{1mm}
\begin{enumerate}[noitemsep]
\vspace{1mm}
\item You can download a compilation of key biogeochemical variables (as \textbf{netCDF} files) from the \textsf{\footnotesize mymuffin} \href{http://www.seao2.info/cgenie/data/GEnIE_observations.nc}{webpage} ('\textit{Observations for ECOGEM}').
\vspace{1mm}
\item You can open the \textsf{\footnotesize GEnIE\_observations.nc} file in \textbf{Panoply} in the same way as you opened the \textsf{\footnotesize fields\_ecogem\_2D.nc} file.
\vspace{1mm}
\item You can now compare the model output to a variety of key biogeochemical variables that have been derived from ocean measurements. The variables in the \textsf{\footnotesize GEnIE\_observations.nc} \textbf{netCDf} file include:
\vspace{1mm}
\begin{itemize}[noitemsep]
\item T and S (not so interesting, except to e.g. evaluate how well \textbf{muffin} simulates different temperature regimes (and habitats)). 
\item DIC and ALK (ignore for the purpose of this chapter).
\item Phosphate is of more interest -- how well does \textbf{muffin} simulate surface nutrient availability, and to what degree does ecosystem complexity help explain observations. 
\item Observed Chlorophyll is remote-sensed, and can be contrasted with the chlorophyll biomass simulated by \textbf{ECOGEM}.
\end{itemize}
\vspace{1mm}
You might visually compare model with observations, and/or e.g. create difference maps.
\vspace{1mm}
\item When doing this, be asking yourself the question: does the model perform well or poorly with respect to reproducing these variables? If not, \uline{why not}?
\end{enumerate}
\vspace{2mm}

%------------------------------------------------
\newpage
%------------------------------------------------

\section{Ecosystem configuration}

In the last section you ran a very simple configuration of the \textbf{ECOGEM} ecosystem model, and compared it to observations. In this section we are going to add a bit more ecological realism, with the aim of improving model performance (i.e. as contrasted against observations). We will start by adding a zooplankton population that  should bring a degree of `top-down' control to the phytoplankton population.
\vspace{2mm}

Details of the ecosystem are specified in the \textit{user-config} file: \textsf{\footnotesize EXP.8.1}

\begin{enumerate}[noitemsep]

\vspace{1mm}
\item The \textit{user-config} file can be used to configure the ecological model to your liking. 

\vspace{1mm}
One of the most important amendments to note straight away can be seen near the very start:
\\\texttt{bg\_par\_bio\_prodopt="NONE"}. This effectively disables the simple biological export scheme in \textbf{BIOGEM}, replacing it with the explicit biology of \textbf{ECOGEM}. This is a necessary step whenever running \textbf{ECOGEM}, because we do not want the implicit and explicit biological schemes to be implemented simultaneously ...
\vspace{1mm}
\item We can also see a load of other model parameters. Any that begin with `\texttt{bg\_}' correspond to \textbf{BIOGEM}, while `\texttt{eg\_}' corresponds to \textbf{ECOGEM}. The \textbf{ECOGEM} parameters begin further down the file.
\vspace{1mm}
\item One of the most important parameters specifies the \textit{ecosystem configuration} file:
\vspace{0mm}
\small\begin{verbatim}
eg_par_ecogem_plankton_file ='NPD.eco'
\end{verbatim}\normalsize
\vspace{-0mm}
This points to a file (located in \textsf{\footnotesize \textasciitilde{}/cgenie.muffin/genie-ecogem/data/input/}) that specifies every plankton population ('species', if you like) that is accounted for in the model experiment. 

\vspace{1mm}
If you open that file in a text editor, you will see something akin to the following:
\scriptsize\begin{verbatim}
 01                 02    03
 \/                 \/    \/

-START-OF-DATA-
 Phytoplankton    10.00   1
-END-OF-DATA-

 /\                 /\    /\
 01                 02    03

DATA FORMAT AND ORDER
---------------------

COLUMN #01: plankton functional type name
COLUMN #02: plankton diameter (micrometers)
COLUMN #03: number of randomised replicates

INFO: TRACER ASSIGNMENT RULES
-----------------------------
Plankton functional type one of: Prochlorococcus
                                 Synechococcus
                                 Picoeukaryote
                                 Diatom
                                 Coccolithophore
                                 Diazotroph
                                 Phytoplankton
                                 Zooplankton
                                 Mixotroph
\end{verbatim}\normalsize

%------------------------------------------------
\newpage
%------------------------------------------------

\vspace{1mm}
\item The first thing to note is that only the lines in between the \texttt{\small -START-OF-DATA-} and \texttt{\small -END-OF-DATA-} tags are read by the computer. The rest is there solely for your guidance.

\vspace{1mm}
Each line that is entered in the computer-readable area tells the model to create a distinct plankton population in the model. The 'plankton functional type' of this population is specified in the first column, while the plankton diameter is specified in the second column. A '\texttt{1}' must always be placed in the third column\footnote{It doesn't 'do' anything, but the model still needs it ...  why ... WTF?}.

\vspace{1mm}
In this 'NPD' (single nutrient-plankton-detritus) configuration, we only have a 10 micron generic phytoplankton. The ecological and physiological traits of this population are assigned automatically according to the size and the functional type (here: photo-autotroph).

\vspace{1mm}
\item[NOTE:] The only PFTs available at the moment are \texttt{Phytoplankton}, \texttt{Zooplankton} and \texttt{Mixotroph}. The other groups currently have no real functionality associated with them.

\vspace{1mm}
\item[NOTE:] This file format is fussy, and you\uline{ cannot have any empty lines} in between the  \texttt{\small -START-OF-DATA-} and \texttt{\small -END-OF-DATA-} tags -- every line between these tags must have data (3 parameter values).

\vspace{2mm}
\item We can increase the ecological complexity of the model by adding another plankton population. Save the \textit{ecosystem configuration} file under a new and highly intuitive name (such as \textsf{\footnotesize NP\underline{Z}D.eco}), and add another line specifying a 100 micron zooplankton. e.g.:
\small\begin{verbatim}
 Zooplankton     100.00   1
\end{verbatim}\normalsize
It is important that the zooplankton is 10 times larger than the phytoplankton in terms of diameter. This is the optimal predator-prey length ratio in the default configuration. (You could maybe think about changing this value later on.)

\vspace{2mm}
\item  To run the model with this new configuration, change the name of the \textit{ecosystem configuration file} in the \textit{user-config} file\dots
\small\begin{verbatim}
eg_par_ecogem_plankton_file ='NPZD.eco'
\end{verbatim}\normalsize

\vspace{1mm}
\item Save the new \textit{user-config} file under a different name (e.g. \textsf{\footnotesize EXP.8.1.NPZD}). You can now execute the model at the command line\footnote{Don't forget to change the name of the \textit{user-config} file here as well ...} ...
\vspace{-1mm}\small\begin{verbatim}
$ ./runmuffin.sh muffin.CBE.p_worjh2.BASESFeTDTL LABS EXP.8.1.NPDZ 10 
   muffin.CBE.p_worjh2.BASESFeTDTL.FeMIP.SPIN
\end{verbatim}\normalsize\vspace{-1mm}
... or submit it to the cluster queue ...

\vspace{1mm}
\item Once you have completed the new simulation, compare the new results to the old simulation, and also in terms of its ability to reproduce observations. Has the addition of zooplankton to the model improved its behaviour or not? Look also at the global distributions of carbon biomass in the phytoplankton and zooplankton populations (again, a log scale might help).

\vspace{1mm}
\textit{How have the zooplankton interacted with the phytoplankton to change the ecological dynamics in the model?}

\vspace{1mm}
Note that 10 years is clearly far far too short to accommodate any change in the global cycle of nutrients (can carbon and oxygen), but may be sufficient to see much of the impact of changing the assumed ecosystem structure. (You can always run for longer to judge for yourself on what time-scales what components of the Earth system adjust and hence what the 'ideal' (practical) run-time for changing the ecosystem structure might be.)

\end{enumerate}

%------------------------------------------------
\vspace{1mm}
\noindent\rule{4cm}{0.1mm}
%------------------------------------------------

%------------------------------------------------
\newpage
%------------------------------------------------

\subsubsection{Visualising composite data}
\vspace{1mm}

We can perhaps get a better handle on this question by looking at the ratio of phytoplankton-to-zooplankton biomass. Such ratios can, however, be difficult to assess simply by eye-balling two maps. Instead we can use \textbf{Panoply} to combine data arrays.

\vspace{1mm}
\begin{enumerate}[noitemsep]
\vspace{1mm}
\item First close all your \textbf{Panoply} plot windows. Then open a new one for \textsf{\footnotesize C~Biomass - Popn.~001 (10.00~micron phytoplankton)}. Next, select \textsf{\footnotesize C~Biomass - Popn.~002 (100.00~micron zooplankton)}, and click the \textsf{\footnotesize Combine Plot} icon at the top of the \textbf{Panoply} window.
\vspace{1mm}
\item A box will open up asking you \textsf{\footnotesize In which existing plot should I combine the variable}. As you now only have one plot available, this should be a straightforward choice. Click \textsf{\footnotesize Combine}.
\vspace{1mm}
\item A new map should appear showing the total zooplankton carbon biomass minus the total phytoplankton carbon biomass (see the label on the colour scale). This is not what we want. Below the map, under the \textsf{\footnotesize Array(s)} tab, there is a drop down menu showing the range of different ways the two arrays can be combined. We want to look at the Z:P biomass ratio, so select \textsf{\footnotesize Array 2 / Array 1}.
\vspace{1mm}
\item You now need to make sure that you are looking at the right year (you can time-lock the two arrays by clicking on the chain icon). You may also find it helpful to look at the data on a log scale, with a scale range of \textsf{\footnotesize 0.1} to \textsf{\footnotesize 10}. You might also like to change the \textsf{\footnotesize Color Table:} option to \textsf{\footnotesize GMT\_polar.cpt}.
\end{enumerate}

\vspace{2mm}
\noindent \textbf{Questions:}
\vspace{1mm}
\begin{itemize}
\item What does this plot say about the relationship of zooplankton and phytoplankton in different regions of the ocean?
\item In what regions do zooplankton or phytoplankton dominate?
\item What affect does a high Z:P ratio have on the biomass of the phytoplankton population?\footnote{For example, in terms of the chlorophyll concentration.}
\end{itemize}

%------------------------------------------------
\newpage
%------------------------------------------------

\section{Increasing ecological complexity}

In the last section, we looked at the results of some simulations based on `NPD' (one nutrient-(phyto)plankton-detritus) and `NPZD' (one nutrient-(phyto)plankton-zooplankton-detritus) type ecosystem models. Here we will begin to incorporate a bit more ecological complexity.

%------------------------------------------------
\vspace{1mm}
\noindent\rule{4cm}{0.1mm}
%------------------------------------------------

\subsubsection{Plankton size classes}
\vspace{1mm}

We are going to add a few more plankton size classes, so that we end up with small, medium and large phytoplankton and (small, medium and large) zooplankton.

\begin{enumerate}[noitemsep]
\vspace{1mm}
\item Save the \textit{ecosystem configuration} file under a new name (e.g. \textsf{\footnotesize 3P3Z.eco}), replacing the existing plankton populations with the ones described in Table~\ref{planktonconfig1}.
\vspace{1mm}
\item  To run the model with this new configuration, change the name of the \textit{ecosystem configuration file} in the \textit{user-config} file:
\vspace{-2mm}\begin{verbatim}
eg_par_ecogem_plankton_file='3P3Z.eco'
\end{verbatim}\vspace{-2mm}
\vspace{1mm}
\item Save the new \textit{user-config} file under a different name (e.g. \texttt{BSS.3P3Z.SPIN}) and then run the new model at the command line (e.g. for 10 years).
\end{enumerate}
\vspace{2mm}

\vspace{-2mm}
\begin{table}[htp!]
\begin{center}
\caption{Plankton functional groups and sizes.}
\begin{tabular}{rlc}
\hline
$j$     & PFT                   & \multicolumn{1}{r}{Diameter ($\mu$m)}  \\
\hline
1       & Phytoplankton         & 0.6  \\
2       & Phytoplankton         & 6.0  \\
3       & Phytoplankton         & 60.0  \\
\hline
\end{tabular}
\begin{tabular}{rlc}
\hline
$j$     & Functional Type       & \multicolumn{1}{r}{Diameter ($\mu$m)}  \\
\hline
4       & Zooplankton           & 6.0  \\
5       & Zooplankton           & 60.0  \\
6       & Zooplankton           & 600.0  \\
\hline
\end{tabular}
\label{planktonconfig1}
\end{center}
\end{table}
\vspace{-6mm}

%------------------------------------------------
\vspace{1mm}
\noindent\rule{4cm}{0.1mm}
%------------------------------------------------

\subsubsection{Viewing 2D time-slice output}
\vspace{1mm}

Open up the 2D \textit{time-slice} data for the new experiment, following the same procedure as in the previous section. You will now see a lot more \textit{time-slice} variables now appear listed in \textbf{Panoply}. We have all the same diagnostics as before, plus some new ones relating to the new plankton populations you have just added. There are also variables describing the size distribution and diversity of the photosynthetic community (non-phototrophic populations are ignored in these metrics). These were not included before, because there was only one phytoplankton population.

\vspace{2mm}
\begin{itemize}[noitemsep]

\item[\textbf{Size fractions}] \
\\Variables  ``\textsf{\footnotesize eco2D\_Size\_Frac\_...}'' give the chlorophyll biomass in the three size fractions:
\vspace{1mm}
\begin{enumerate}[noitemsep]
\item picophytoplankton (diameter $\le$ 2 $\mu$m)
\item nanophytoplankton (2 $<$ diameter $\le$ 20 $\mu$m)
\item microphytoplankton (diameter $>$ 20 $\mu$m)
\end{enumerate}
\vspace{1mm}

%------------------------------------------------
\newpage
%------------------------------------------------

\item[\textbf{Size metrics}] \
\\Variables  '\textsf{\footnotesize eco2D\_Size\_...}' give metrics describing the {phytoplankton} size distribution.
\vspace{1mm}
\begin{itemize}[noitemsep]
\item \textsf{\footnotesize eco2D\_Size\_Mean}: Geometric mean\footnote{ We use the geometric mean and standard deviation, because phytoplankton biomass is approximately log-normally distributed across the phytoplankton size range.} phytoplankton diameter, weighted by carbon biomass
\item \textsf{\footnotesize eco2D\_Size\_Stdev}: Geometric standard deviation\footnote{ The geometric standard deviation describes the \textit{relative} size range of the phytoplankton. For a geometric standard deviation of $\sigma$, $\sim$68.2\% of the phytoplankton carbon biomass will be in cells no more than $\sigma$ orders of magnitude smaller or larger than the geometric mean size.} of phytoplankton diameter, weighted by carbon biomass.
\item \textsf{\footnotesize eco2D\_Size\_Minimum}: Diameter of smallest phytoplankton contributing $>$0.1\% of the total phytoplankton carbon biomass.
\item \textsf{\footnotesize eco2D\_Size\_Maximum}: Diameter of largest phytoplankton contributing $>$0.1\% of the total phytoplankton carbon biomass.
\end{itemize}
\vspace{1mm}

\item[\textbf{Diversity metrics}] \
\\Variables  '\textsf{\footnotesize eco2D\_Diversity\_...}' give metrics describing the {phytoplankton} diversity.\footnote{NOTE: The threshold index is a fairly crude measure of the total number of species in the community, relative to a small and arbitrary threshold of relative biomass. This index is not very sensitive to the relative biomass of individual species (although one very successful species can raise the absolute value of the threshold, thus lowering the diversity).
\\The other three indices do more to quantify the evenness of the community. The more unequal the proportional abundances, the smaller the value of the index. If almost all the abundance is concentrated into one type and all the other types are very rare, the latter three indices can become very small. A community with fewer species, but with more evenly distributed biomass, may well have higher values for these three diversity indices.}
\vspace{1mm}
\begin{itemize}[noitemsep]
\item \textsf{\footnotesize eco2D\_Diversity\_Threshold}: the threshold diversity index. The number of species contributing $>$0.1\% of the total phytoplankton carbon biomass [\textit{Barton et al.}, 2010]\footnote{A D Barton, S Dutkiewicz, G Flierl, J Bragg, and M Follows. Patterns of diversity in marine phytoplankton. \textit{Science}, 327(5972):1509–1511, 2010.}.
\item \textsf{\footnotesize eco2D\_Diversity\_Berger}: the inverse Berger(-Parker) index  [\textit{Berger and Parker}, 1970]\footnote{W H Berger and F L Parker. Diversity of planktonic foraminifera in deep-sea sediments. \textit{Science}, 168 (3937):1345–1347, 1970.}. The proportion of carbon biomass made up by all but the single most dominant population. For example, if the dominant population accounts for 40\% of the total carbon biomass, inverse Berger (-Parker) index is 0.6.
\item \textsf{\footnotesize eco2D\_Diversity\_Simpson}: the inverse (Gini-)Simpson index [\textit{Simpson}, 1949]\footnote{E H Simpson. Measurement of diversity. \textit{Nature}, 163:688, 1949.}. This is effectively the probability that two samples taken at random from the community will be from a different species (note that the probability of selecting a population is dependent on carbon biomass, not cell abundance). If we define the proportional biomass of each species as its relative contribution to the total carbon biomass in the community, the inverse Gini(-Simpson) index is calculated as one minus the sum of the squares of the proportional biomasses of each species.
\item \textsf{\footnotesize eco2D\_Diversity\_Shannon}: the Shannon(-Wiener or -Weaver) index  [\textit{Shannon}, 1948]\footnote{C E Shannon. A mathematical theory of communication. \textit{The Bell System Technical Journal}, 27(379-423 and 623-656), 1948.}. With the proportional biomass defined as above, the Shannon index is defined as the sum of [the proportional biomass multiplied by the logarithm of the proportional biomass] for each species.
\end{itemize}
\vspace{1mm}

\end{itemize}
\vspace{1mm}

Have a look at some of these metrics, but bear in mind that they summarise the diversity of a phytoplankton community that includes just three species. They are probably not that revealing, so we will come back to them later. Instead, have a look some of the other metrics describing the model ecosystem.

\vspace{2mm}
\noindent \textbf{Questions:}
\vspace{1mm}
\begin{itemize}
\item What are the global distributions of the different size classes?
\item How do the global biomass distributions compare to variables such as temperature\footnote{Ocean temperature is saved in \textsf{\scriptsize fields\_biogem\_3D.nc}}, or primary production (\textsf{\footnotesize Uptake Fluxes C})?
\item How does nutrient, light and temperature limitation vary between the size classes?
\item Can you account for the distribution biomass between the size classes according to the different limiting factors?
\end{itemize}
\vspace{2mm}

%------------------------------------------------
\vspace{1mm}
\noindent\rule{4cm}{0.1mm}
%------------------------------------------------

\subsubsection{Create your own ecosystem}
\vspace{1mm}

Save the \textit{ecosystem configuration} file under a new name and add some more plankton populations (whatever and as many as you like\footnote{Just bear in mind that the more populations you put in, the slower the model will run!}). Update the \textbf{muffin} parameter \texttt{eg\_par\_ecogem\_plankton\_file} in the \textit{user-config} and save this file under a new name.  Run the new model.

\vspace{2mm}
\noindent \textbf{Questions:}
\vspace{1mm}
\begin{itemize}
\item How many populations can you get to coexist (i.e. each having a non trivial ($>$0.1\%) biomass)?
\item What effect do the new populations have on the community as a whole?
\item What effect, if any, do your additions have on the strength of biological export?\\(e.g. look at \textsf{\footnotesize bio\_fpart\_POC} in \textsf{\footnotesize fields\_biogem\_3D.nc}.)
\end{itemize}
\vspace{2mm}

%------------------------------------------------
\newpage
%------------------------------------------------

\section{Build it up, tear it down}
\vspace{1mm}

\subsubsection{A fully size-structured  ecosystem}
\vspace{1mm}

We are now going to switch to a more diverse version of the size-structured ecosystem model. This configuration has 8 size classes of phytoplankton, and 8 size classes of zooplankton, as shown in Table~\ref{planktonconfig2}.

\begin{enumerate}[noitemsep]
\vspace{1mm}
\item Save the \textit{ecosystem configuration} file under a new name, replacing the existing plankton populations with the ones described in Table~\ref{planktonconfig2}.
\vspace{1mm}
\item Create a new \textit{user-config} and point to the new \textit{ecosystem configuration} file.
\vspace{1mm}
\item Run the new model for at least 20 years (this will probably take about 10 minutes).
\end{enumerate}

\vspace{-0mm}
\begin{table}[htp!]
\begin{center}
\caption{Plankton functional groups and sizes.}
\begin{tabular}{rlc}
\hline
$j$     & PFT                   & \multicolumn{1}{r}{Diameter ($\mu$m)}  \\
\hline
1       & Phytoplankton         & 0.2  \\
2       & Phytoplankton         & 0.6  \\
3       & Phytoplankton         & 1.9  \\
4       & Phytoplankton         & 6.0  \\
5       & Phytoplankton         & 19. 0 \\
6       & Phytoplankton         & 60.0  \\
7       & Phytoplankton         & 190.0  \\
8       & Phytoplankton         & 600.0  \\
\hline
\end{tabular}
\begin{tabular}{rlc}
\hline
$j$     & Functional Type       & \multicolumn{1}{r}{Diameter ($\mu$m)}  \\
\hline
9       & Zooplankton           & 0.6  \\
10      & Zooplankton           & 1.9  \\
11      & Zooplankton           & 6.0  \\
12      & Zooplankton           & 19.0 \\
13      & Zooplankton           & 60.0  \\
14      & Zooplankton           & 190.0  \\
15      & Zooplankton           & 600.0  \\
16      & Zooplankton           & 1900.0  \\
\hline
\end{tabular}
\label{planktonconfig2}
\end{center}
\end{table}
\vspace{-6mm}

\vspace{1mm}
\noindent\rule{4cm}{0.1mm}
%------------------------------------------------

\subsubsection{Ecosystem characteristics}
\vspace{1mm}

We can now begin to look at the size and diversity metrics  in a more meaningful way.

\begin{enumerate}[noitemsep]
\vspace{1mm}
\item Look first at:
\vspace{1mm}
\begin{enumerate}[noitemsep]
\item the total carbon biomass
\item the carbon uptake flux (i.e. primary production)
\item the geometric mean size
\end{enumerate}
\vspace{1mm}
Make sure that in each case that you are looking at the \uline{last} year of model output. You may also find it useful to adjust the colour scale, or to change to a logarithmic colour scale (e.g. try a logarithmic scale from 2 to 20 microns for the geometric mean size).

\vspace{1mm}
Looking at the maps, we can perhaps pick out three different ``biomes'' in terms of their community properties:
\vspace{1mm}
\begin{enumerate}
\item The low-latitude oligotrophic gyres are relatively unproductive, and support some of the lowest annual mean biomass in the surface ocean. In these regions the mean phytoplankton size is very small.
\item Subpolar latitudes between 40$^\circ$ and 50$^\circ$ (either N or S) are much more productive, and support very high annual mean biomass. These communities also have the highest mean sizes of any region.
\item The polar oceans are also highly productive (except perhaps the high Arctic), and support relatively high annual mean biomass. These communities are made up (in the model, at least) of slightly smaller phytoplankton than we see in the subpolar regions.
\end{enumerate}

%------------------------------------------------
\newpage
%------------------------------------------------
%
\vspace{1mm}
\item What can we find out about the community structure in these regions? Open up some of the other metrics describing the community (standard deviation of size distribution, size fractionation, diversity, limiting factors). What can you find out about the community structure within each region, in terms of coexistence and exclusion?
\vspace{1mm}
\begin{itemize}
\item Does the community span a broad or narrow size range?
\item How many size classes are coexisting in each biome?
\item What is the smallest and largest size class in each biome?
\item How much biomass is concentrated in each size fraction (picoplankton, nanoplankton and microplankton)?
\end{itemize}
\end{enumerate}
\vspace{1mm}

Overall -- what factors do you think are most important in terms of dictating the global distribution of each size class?

To find out the answers to these questions, you are going to pull the model apart, and then put it back together. At each stage the aim is to bring in a different limiting factor, so that you can see its effect on the model behaviour.

\vspace{1mm}
\noindent\rule{4cm}{0.1mm}
%------------------------------------------------

\subsubsection*{The fundamental niche}
\vspace{1mm}

The first step is to find out the impact of abiotic factors on the distribution of different phytoplankton sizes. In other words, we need to find out what the distribution of the phytoplankton would be in the absence of any ecological interactions, such as resource competition and predation. This is effectively their `fundamental niche'.

\vspace{1mm}
The fundamental niche is fairly abstract, and not something that can be measured in the real world. In model world, however, we can get a useful estimate of the fundamental niche by making a few simple changes to the model.

\begin{enumerate}[noitemsep]
\vspace{1mm}
\item First of all, you can remove all predation, simply by removing the zooplankton from the \textit{ecosystem configuration} file.
\vspace{1mm}
\item Next, you also need to remove all competition for nutrients and light. This involves tweaking the model equations so that the phytoplankton are not nutrient limited, and do not attenuate light. To do this, all you need is to add the following line to the \textit{user-config} file:
\vspace{-1mm}
\small\begin{verbatim}
eg_fundamental = .TRUE.
\end{verbatim}\normalsize
\vspace{-1mm}
\item If you now run the model you should have a community of eight phytoplankton size classes that are growing solely as a function of the incoming light and the temperature. This growth will be balanced balanced the basal (i.e. non-grazing) mortality. As there is no feedback between the ecosystem and the environment, populations that can survive will grow exponentially and without limit, potentially reaching astronomical abundance in very little time. Populations that cannot survive will rapidly decline to almost nothing.
\vspace{1mm}
\\The regions in which each plankton shows positive growth defines its fundamental niche. This is a function of abiotic conditions only, and is the absolute limit of its geographical range. Look at the carbon biomass distribution in each size class (set the data range in each case from \(0\) to \(1mmolCm^{-3}\)).
\vspace{1mm}
\begin{itemize}
\item How and why does the fundamental niche vary with size?
\item Could the limits of the fundamental niche explain some of the patterns seen in the full model?
\end{itemize}

\end{enumerate}
\vspace{2mm}

\vspace{1mm}
\noindent\rule{4cm}{0.1mm}
%------------------------------------------------

%------------------------------------------------
\newpage
%------------------------------------------------
%
\subsubsection*{Resource competition.}
\vspace{1mm}

The next step is asses the impact of resource competition. We are first going to do this in the absence of any zooplankton grazing.

\begin{enumerate}[noitemsep]
\vspace{1mm}
\item All you need to do at this stage is to re-enable nutrient and light competition. To do this, simply delete `\texttt{eg\_fundamental~=~.TRUE.}' from the \textit{user-config} file (or comment out the line to disable it), and save under a new name. Leave the \textit{ecosystem configuration} file as it is.

\vspace{1mm}
\item You should have a community of eight phytoplankton size classes that are competing for nutrients and light, again as a function of temperature. This is a much more realistic simulation, as feedbacks between the ecosystem and the environment serve to limit the size of the phytoplankton populations. Examine the model to find out:
\vspace{1mm}
\begin{itemize}
\item What size classes are able to persist when resource competition is enabled?
\item Why are different size classes more or less abundant in different areas?
\item How does the distribution of each size class compare to the fundamental niche?
\item What are the reasons for any differences?
\end{itemize}
\end{enumerate}
\vspace{1mm}

Phytoplankton biogeography at this stage begins to approximate the realised niche, which defines the range of conditions that support a population in the presence of ecological interactions. Note that at this stage, however, we have ignored the effects of any predator-prey interactions, as the zooplankton grazers are still missing.

\vspace{1mm}
\noindent\rule{4cm}{0.1mm}
%------------------------------------------------

\subsubsection*{Resource competition + one generalist zooplankton}
\vspace{1mm}

The previous simulation is clearly unrealistic (although, hopefully, informative). You are now going to add back in just a single zooplankton class, that grazes equally on all plankton (including itself).

\begin{enumerate}[noitemsep]
\vspace{1mm}
\item Add a 100 micron zooplankton into the \textit{ecosystem configuration} file, and save under a new name. Also update the \textit{user-config} file to reflect the change, and save under a similar name.
\vspace{1mm}
\item You need to modify the model so that the zooplankton eats all prey with equal preference. This can be done by adding the following lines to the \textit{ecosystem configuration} file.
\vspace{-1mm}\small\begin{verbatim}
eg_ns=1
eg_pp_sig_a=1.0e99
\end{verbatim}\normalsize\vspace{-1mm}
\item[NOTE:] For aficionados, the first parameter disables prey-switching (i.e. predators no longer preferentially attack the most abundant prey). The second parameter increases the width of the grazing kernel (i.e. predators can attack a range of prey across a huge size range with equal preference).
\vspace{1mm}
\item The addition of zooplankton to the model community should give a more accurate approximation of the realised niche.
\vspace{1mm}
\begin{itemize}
\item Does the addition of a single zooplankton grazer enable more or less coexistence?
\item What factors might be responsible for any shifts in biogeography?
\end{itemize}
\end{enumerate}

\vspace{1mm}
\noindent\rule{4cm}{0.1mm}
%------------------------------------------------

%------------------------------------------------
\newpage
%------------------------------------------------
%
\subsubsection*{Resource competition + one ``switching'' zooplankton}
\vspace{1mm}

You began with a full food-web containing 8 phytoplankton and 8 zooplankton size classes. The diversity of zooplankton clearly has an effect on the phytoplankton community that is not seen in the previous experiment. This effect can be imitated with just one generalist zooplankton if we instruct it to graze preferentially on the most successful prey.

Re-enable this 'prey switching' effect by changing the following control parameter to a \texttt{2}:
\vspace{-1mm}\small\begin{verbatim}
eg_ns=2
eg_pp_sig_a=1.0e99
\end{verbatim}\normalsize\vspace{-1mm}
Compare this simulation to the first experiment (8 phytoplankton and 8 zooplankton) to see how the inclusion of prey switching increases coexistence through the `kill-the-winner' mechanism.

\vspace{1mm}
\begin{itemize}
\item How does nutrient limitation change with phytoplankton size, and how might zooplankton be affecting this?
\item Look at the C:P biomass ratio in the community as a whole, and compare to your estimates from the NPZD model (Lesson~1).
\item How does the C:P ratio vary with size? How does having a diverse community affect the coupling of carbon and limiting nutrients?
\end{itemize}

\vspace{1mm}
\noindent\rule{4cm}{0.1mm}
%------------------------------------------------

\subsection*{Further questions to answer}
\vspace{1mm}

\begin{itemize}
\item What sets the fundamental niche, and how does it change with size?
\item How is the fundamental niche modified by resource competition?
\item What species are favoured in terms of nutrient competition?
\item How is the outcome of competition affected by...
\begin{itemize}
\item Abiotic conditions?
\item Increased mortality (through generalist grazing)?
\item Density-dependent mortality (through specialist grazing)?
\end{itemize}
\item Do these experiments tell you all you need to know?\\What other modifications can you think of making?
\end{itemize}

%------------------------------------------------
\newpage
%------------------------------------------------

\section{Mixotrophy}

Try adding some mixotrophs to the phytoplankton and zooplankton already present in the community. These will have exactly half the nutrient uptake traits of phytoplankton of a similar size, and half the prey capture traits of zooplankton if a similar size. To do this:

\begin{enumerate}[noitemsep]
\vspace{1mm}
\item Save the previous \textit{ecosystem configuration} file under a new name.
\vspace{1mm}
\item Edit the new \textit{ecosystem configuration} file and add an additional line to add a mixotroph:
\vspace{-1mm}\small\begin{verbatim}
 Mixotroph    xxx   1
\end{verbatim}\normalsize\vspace{-1mm}
where \texttt{xxx} is the class size of the mixotroph. (And remember to save it.)
\vspace{1mm}
\item Update the \textit{user-config} to point to the new \textit{ecosystem configuration} file, and save again under a new name. (It is generally a good idea to make a note of the name and goal of each experiment as you set it up.)
\vspace{1mm}
\item Run the new model for at least 10-20 years.
\end{enumerate}
\vspace{1mm}

\noindent Further questions to explore/answer:

\vspace{1mm}
\begin{itemize}
\item How does this effect the mean and standard deviation of cell size?\\(Size and diversity metrics will be calculated for phytoplankton and mixotrophs together)
\item How does mixotrophy effect the C:P ratio of organic matter?
\item How does the realised niches of mixotrophs compare to the fundamental niches of phytoplankton?
\end{itemize}

\vspace{1mm}
\noindent Also: try replacing all the phytoplankton and zooplankton with a range of sizes of mixotrophs. How does the simulation differ from one with the same size range of sperate phytoplankton and zooplankton classes?

For instance -- you might also try having an ecosystem comprising just a single small (e.g. \(10 \mu m\) mixotroph (no phytoplankton and no zooplankton), and perhaps an ecosystem with one small mixotroph \(10 \mu m\) and one additional mixotroph, 10 times lager \(100 \mu m\). Compare the productivity of such an ocean compared to some of your previous simplified ecosystem configurations (such as a single small \(10 \mu m\) phytoplankton, and a configuration with a single small phytoplankton and a single larger \(100 \mu m\) grazer).
%
%------------------------------------------------
\newpage
%------------------------------------------------

\section{Ecology in ... future oceans}

What might happen to primary production and the biological pump as (future) climate continues to change? What might happen to the distribution of species (here: size classes) in the ocean? Does it 'matter' e.g. for marine bio geochemical cycling and feedbacks on atmospheric \(pCO_{2}\)? (And how different is the Earth system response when using a complex ecosystem model rather than the highly simplified representation of biological export you used previously?)

\vspace{1mm}
There are all good questions for a model!

\vspace{1mm}
The best place to start is with the full complexity ecosystem structure with 8 size classes for both phytoplankton and zooplankton (as described in \textit{Ward et al.} [2018]), and continuing on from a \textit{re-start} of the pre-industrial ocean (and ecology): \textsf{\footnotesize muffin.CBE.p\_worjh2.BASESFeTDTL.FeMIP.SPIN}. For the \textit{user-config} -- you can either utilize one of the ones you created in earlier ecosystem experimenting, or adapt the  configuration used to create the \textit{re-start} \footnote{Remember to rename it so you do not overwrite the \textit{re-start} ...} -- \textsf{\footnotesize muffin.CBE.p\_worjh2.BASESFeTDTL.FeMIP.SPIN} which you can find in: \textsf{\footnotesize user-configs/EXAMPLES}. (The \textit{base-config} is as per you have been using in this Chapter -- \textsf{\footnotesize muffin.CBE.p\_worjh2.BASESFeTDTL}.) 
\vspace{2mm}

Then:
\vspace{1mm}

\begin{enumerate}[noitemsep]

\vspace{1mm}
\item First, create a \textit{control} based on your chosen \textit{user-config} so that you can more easily spot any mis-matched parameter values or other problems -- do to this, turn off the atmospheric \(pCO_{2}\) \textit{forcing} (so that \(pCO_{2}\) can respond to and indicate any significant changes in the marine carbon cycle), but leave on the dust flux \textit{forcing} (because we still need a supply of dissolved iron to the ocean surface):
\vspace{-1mm}\small\begin{verbatim}
# specify forcings
bg_par_forcing_name="worjh2.FeMahowald2006"
\end{verbatim}\normalsize\vspace{-1mm}

\vspace{1mm}
\item Second, you will need either:

\vspace{1mm}
\begin{itemize}[noitemsep]
\item A \textit{user-config} with an conceptual emissions scenario (e.g. anything that you have played with previously). This can be run directly on from the pre-industrial spin-up.

\vspace{1mm}
This will allow you to explore questions concerning the impact on marine ecosystems and carbon cycling of the rate of warming (driven by your assumed rate of carbon emissions), and/or the importance of and sensitivity of marine ecosystems and carbon cycling, to the total (maximum) warming.

\vspace{1mm}
A template emissions \textit{forcing} (plus dust) is provided as:
\vspace{1mm}
\\\textsf{\footnotesize worjh2.FeMahowald2006.FpCO2\_Fp13CO2}

\vspace{1mm}
As per before, you need to re-scale the values in the forcing because the \(CO_{2}\) flux is assumed to be in units of \(PgC\:yr^{-1}\) \footnote{As per notes in the \textit{forcing} \textit{time-series} \textsf{\footnotesize .sig} file itself.} (and \(\delta^{13}C\) is given a default value of \(1.0\)). i.e. add the lines:
\vspace{-1mm}\small\begin{verbatim}
bg_par_atm_force_scale_val_3=8.3333e+013
bg_par_atm_force_scale_val_4=-27.0
\end{verbatim}\normalsize\vspace{-1mm}
and remember that the forcing is provided as a pulse of \(1\:molyr^{-1}\) (\(1PgC \:yr^{-1}\) when scaled) carbon emissions over just a single year, and hence the \textit{time-series} file for the \textit{forcing}: \textsf{\footnotesize biogem\_force\_flux\_atm\_pCO2\_sig.dat} needs to be edited -- see Chapter \ref{ch:fossil-fuel-co2}.

\end{itemize}

%------------------------------------------------
\newpage
%------------------------------------------------
Or:
%\vspace{1mm}

\begin{itemize}[noitemsep]
\item A historical transient (plus dust) \textit{forcing} and then run from e.g. 1750 through 2010 (as per before), for which the forcing is:
\vspace{1mm}
\\\textsf{\footnotesize worjh2.FeMahowald2006.historical2010}

\vspace{1mm}

\vspace{1mm}
Remember that here that you do not re-scale the atmospheric \(pCO_{2}\) (or \(\delta^{13}C\)) value as was done with the forcing used to generate the re-start (this historical transient forcing includes the actual values of  \(pCO_{2}\) and in the correct units). 

\vspace{1mm}
Note that with the full complexity ecosystem model included, the 245 years of this experiment will take much longer than before when you ran a historical transient experiment. You could 'cheat' a little and start from year 1850 rather than 1765, on the basis that nothing much happens in between (in terms of rising \(pCO_{2}\) and warming). To do this -- simply run the transient experiment only for 160 years (1850 to 2010) and adjust the starting date in the \textit{user-config}:
\vspace{-1mm}\small\begin{verbatim}
# change start year
bg_par_misc_t_start=1850.0
\end{verbatim}\normalsize\vspace{-1mm}

\vspace{1mm}
You will also need \textit{time-series} and \textit{time-slice} saving that aligns with the historical (and future) period:
\vspace{-1mm}\small\begin{verbatim}
# save frequency
bg_par_infile_slice_name='save_timeslice_historicalfuture.dat'
bg_par_infile_sig_name='save_timeseries_historicalfuture.dat'
\end{verbatim}\normalsize\vspace{-1mm}
(i.e. just as before when you ran a historical transient).

\vspace{1mm}
Once you have run this experiment, you can use it as a \textit{re-start} for a future emissions scenario, which could e.g. involve one of:
\vspace{1mm}
\begin{itemize}[noitemsep]
\item \textsf{\footnotesize worjh2.FeMahowald2006.FpCO2\_Fp13CO2\_A1\_AIM}
\item \textsf{\footnotesize worjh2.FeMahowald2006.FpCO2\_Fp13CO2\_A2\_ASF}
\item \textsf{\footnotesize worjh2.FeMahowald2006.FpCO2\_Fp13CO2\_B1\_IMAGE}
\item \textsf{\footnotesize worjh2.FeMahowald2006.FpCO2\_Fp13CO2\_B2\_MESSAGE}
\end{itemize}
\vspace{1mm}
(or adapt or create your own) and remember to add the emissions units scaling:
\vspace{-1mm}\small\begin{verbatim}
bg_par_atm_force_scale_val_3=8.3333e+013
bg_par_atm_force_scale_val_4=-27.0
\end{verbatim}\normalsize\vspace{-1mm}

\end{itemize}

\vspace{1mm}
\noindent Whichever route you chose, will also require (another) control experiment run for the same number of years (and with dust but not \(pCO_{2}\) \textit{forcing}).

\end{enumerate}

\vspace{1mm}
\noindent What to look for? The questions (at the start) should help guide you in this, and might include:

\vspace{1mm}
\begin{itemize}[noitemsep]
\item Patterns of primary productivity and export (where do they increase, decrease), plus time-series of global totals (what is the global impact of warming?).
\item Patterns of ocean oxygenation (and patterns of dysoxia and anoxia).
\item Surface ocean nutrient distributions, and nutrient limitation.
\item Spatial patterns of plankton size -- either simply for mean size, or look at the patterns of biomass of specific size classes. (You might follow a the preferred habitat/location of a particular size class with latitude to see whether it experiences a 'range shift'.)
\item Diversity etc.
\end{itemize}

\vspace{1mm}
\noindent You might also contrast with emissions scenarios that you might have run previously employing the simplified biological export scheme in \textbf{BIOGEM}, rather than the complex ecosystem model of \textbf{ECOGEM}. (Make sure you are comparing between the same emissions scenarios). The question to be answered would be: does including a complex representation of  marine ecology 'matter'?

%------------------------------------------------
\newpage
%------------------------------------------------

\section{Ecology in ... past oceans}

Here we consider a series of examples\footnote{Either published or in the works or existing in some publication fantasy ...} of the \textbf{ECOGEM} model of marine ecology, applied to published paleo configuration of \textbf{muffin} and used to ask questions of how different could marine carbon cycling (and atmospheric \(pCO_{2}\)) and oxygenation have been in the (relatively recent) past and how do model projections compare with proxy observations.

Three examples come  from the Last Glacial Maximum, with direct comparison being made to post glacial time (here, the late Holocene), with the intention to explore the role of climate cooling, altered ocean circulation, and increased iron supply to the ocean in modifying plankton distributions and size structures. Secondly, from around the time of the Paleocene-Eocene Thermal Maximum (PETM) some 55 Myr ago, now both considering a different paleogeography, a warmer ocean, and then transient warming on top of that. And thirdly, marine ecology and carbon cycling in the aftermath of the impact and extinction event at the end of the Cretaceous, some 66 Myr ago, and return to our earlier exploration of what happens if you lose the larger plankton in the ocean.

%------------------------------------------------
\newpage
%------------------------------------------------
%
\subsection*{The Last Glacial}

The Last Glacial Maximum (LGM) (ca. 19 to 23 ka) was characterized by lower sealevel and higher ocean salinity, colder ocean temperatures, and a reorganized meridional overturning circulation in the Atlantic. The latter 2 changes in particular should have impacted marine ecosystems and indeed, observations suggest range migration (temperature-tracking) and shifts in the zone of highest productivity in the Southern Ocean, amongst other impacts. The aim of this particular investigation, is to assess what these ecological changes are (at least in model world).

Provided is a configuration of LGM ocean circulation that has been tuned to fit observations of benthic carbon isotopes, which provide an observational constrain on large-scale ocean circulation. To this, you'll an ecosystem (in place of the default \textbf{BIOGEM} 'induced export' scheme). Also provided is a \textbf{muffin} configuration for late Holocene ('HOL') (0-6 ka), created in exactly the same way and also tuned to respective (0-6 ka) benthic carbon isotope data. Configuration HOL provides a point of comparison (or control) for you to compare the ecology in a colder, LGM ocean against.\footnote{i.e. you run pairs of HOL and LGM experiments and contrast between them, rather than necessarily comparing to previous modern configurations.}

\vspace{1mm}

The \textit{user-configs} you need to use, or copy-rename, can be found in the directory:
\vspace{1mm}
\\\textsf{\footnotesize genie-userconfigs/MS/odalenetal.CP.2019}
\vspace{1mm}
\\\noindent(and you can run your experiments from here\footnote{If you use them in this directory, make sure at the command line, you replace LABS with \texttt{MS/odalenetal.CP.2019}.}, or better, copy-edit-rename the \textit{user-configs} you run experiments with, from the \textsf{\footnotesize LABS} sub-directory).

Read the \textsf{\footnotesize readme.txt} file for instructions for the basic set of command line parameters needed, but remember that you may be running your \textit{user-config} with a different name and likely from a different sub-directory.
Spin up the following experiments  (and then experiment with them later)\footnote{Noting that the long file-names differ only in a single '\textsf{v}' vs. an '\textsf{a}' ...}:

\vspace{1mm}
\begin{itemize}[noitemsep]
\item[(1)] \textsf{\footnotesize muffin.CB.GIteiiaa.BASESFeTDTL\_rb } \textit{base-config} \\+ \textsf{\footnotesize muffin.CB.GIteiiaa.BASESFeTDTL\_rb.SPIN } \textit{user-config}
\item[(2)] \textsf{\footnotesize muffin.CB.GIteiiva.BASESFeTDTL\_rb } \textit{base-config} \\+ \textsf{\footnotesize muffin.CB.GIteiiva.BASESFeTDTL\_rb.SPIN } \textit{user-config}
\end{itemize}
\vspace{1mm}

\noindent Submit these to the cluster ...  remembering that you will need to recompile \textbf{muffin} between (1) and (2) (and potentially also re-compile before running (1) if you have not  used that particular \textit{base-config} immediately prior). Run the spin-ups for 10,000 years.

Neither of these configurations currently have an ecosystem enabled, but they will provide a baseline against which you can  contrast a pair of experiment that include explicit ecology. What you need to do know is to add in/enable \textbf{ECOGEM}. To do this, you need to modify both \textit{base-} and \textit{user-config} files of both the HOL and LGM model configurations.\footnote{Strongly recommended that that you copy-rename both sets of files and edit the copies.}

\begin{enumerate}[noitemsep]

\vspace{1mm}
\item \textit{base-config}

\vspace{1mm}
First, you need to enable the \textbf{ECOGEM} module.
\vspace{1mm}
\\At the top of the HOL \textit{base-config} file \textsf{\footnotesize muffin.CB.GIteiiaa.BASESFeTDTL\_Crb}\footnote{And similarly for the LGM one.} you will see the line:
\vspace{-2mm}\small\begin{verbatim}
ma_flag_ecogem=.FALSE.
\end{verbatim}\normalsize\vspace{-2mm}
Simply change this to \texttt{.TRUE.}
\vspace{6mm}
\pagebreak

\vspace{1mm}
\item \textit{user-config}

\vspace{1mm}
Next, some deletions and additions are needed in the \textit{user-config} provided (which only has implicit biological export enabled).
\vspace{1mm}
\\In the section of the file under the heading:
\footnotesize\begin{verbatim}
#
# --- BIOLOGICAL NEW PRODUCTION -------------------------------------
#
\end{verbatim}\normalsize
you are going to delete everything there (in that section) and replace it with:
\vspace{-1mm}\small\begin{verbatim}
# biological scheme ID string
bg_par_bio_prodopt="NONE"
\end{verbatim}\normalsize\vspace{-1mm}
The only other thing you need then is a section of code that defines all the \textbf{ECOGEM} ecological parameters.
\vspace{1mm}
\\ Go open file \textsf{\footnotesize muffin.CBE.p\_worjh2.BASESFeTDTL.FeMIP.SPIN}, which you can find in the \textit{user-config} sub-directory \textsf{\footnotesize MS/wardetal.2018}, and you'll see under the heading:
\footnotesize\begin{verbatim}
#
# --- ECOGEM ----------------------------------------------------------
#
\end{verbatim}\normalsize
a long list of parameter settings (down to the next section headed \texttt{DATA SAVING}). Simply copy-paste this entire section (including \texttt{ECOGEM} header lines if you like), anywhere in your \textit{user-config} file. At the very end of the file would do just fine.\footnote{Try and ensure that there is blank line at the very end of the file just in case \textbf{muffin} has any problems reading it in.}

\vspace{1mm}
And lastly ... under:
\footnotesize\begin{verbatim}
#
# --- MISC ----------------------------------------------------------
#
\end{verbatim}\normalsize

(and the end of the section) add the following lines
\vspace{-1mm}\small\begin{verbatim}
# kraus-turner mixed layer scheme on (1) or off (0)
go_imld = 1
# set mixed layer to be only diagnosed (for ECOGEM)
go_ctrl_diagmld=.true.
# add seaice attenuation of PAR
eg_ctrl_PARseaicelimit=.true.
# relative partitioning of C into DOM
eg_par_beta_POCtoDOC=0.75
\end{verbatim}\normalsize\vspace{-1mm}
This enables a 'mixed layer depth' scheme in the ocean circulation model that \textbf{ECOGEM} needs to calculate light limitation during esp. intervals of high latitude / wintertime deep mixing, but then only allows \textbf{ECOGEM} to 'see' the depth and does not allow physical mixing in the ocean. Then limitation of photosynthesis added to sea-ice covered areas. Finally, an adjustment is made ot carbon vs. phosphorous in exported \(POM\).

\vspace{1mm}
This ... should do-it, i.e. you have added the same tuned ecosystem model as described in \textit{Ward et al.} [2018] to your LGM / HOL \textit{user-configs}.

\end{enumerate}

\vspace{1mm}

Now you are ready to run new LGM and HOL experiments, with a full ecosystem in each\footnote{Remembering that presumably both your \textit{base-config} and \textit{user-config} file names are  different as compared to running the non \textbf{ECOGEM} version described in the \textsf{\footnotesize readme}.}.
\vspace{1mm}

%------------------------------------------------
\newpage
%------------------------------------------------
%
Obvious questions to investigate include not only how ecosystems and patterns of biological export may have differed between LGM and HOL, but also how patterns of nutrients (\(PO_{4}\) and \(Fe\)) may have differed ... and also distributions of dissolved \(O_{2}\) in the ocean (and in the interior, rather than across the ocean surface). A water mass ventilation age tracer has also been simulated and the results of this will help in understanding how global circulation patterns differ between the 2 time intervals.

You can also compare between with and without \textbf{ECOGEM} versions as the only thing that changes between pairs of HOL or pairs of LGM experiments is the biological scheme\footnote{Actually, this is not true, as in the \textbf{ECOGEM} enabled experiments, the mixed layer scheme in the coean circulation model is also activated and which has an impact on ocean circulation. You could then create a 3rd set of HOL+LGM experiments, using the basic \textbf{BIOGEM} implicit export biological scheme, but now also setting the \texttt{go\_imld} parameter to a value of \texttt{1}}.

%------------------------------------------------
\newpage
\subsection*{Warm climates of the past}

In this practical we are going to look at the ocean as it \textit{might} have been just over 55 million years ago, at the Paleocene-Eocene Thermal Maximum (PETM) -- in short a lot warmer, and with a somewhat different continental configuration and hence ocean circulation. The exercise is based on a recent 'ECOGENIE' (\textbf{muffin} configured to include \textbf{ECOGEM}) publication\footnote{Wilson, J.D., F.M. Monteiro, D.N. Schmidt, B.A. Ward, and A. Ridgwell, Linking marine plankton ecosystems and climate: A new modeling approach to the warm early Eocene climate, Paleoceanography and Paleoclimatology, 33, 1439–1452, DOI: 10.1029/2018PA003374 (2018).} which you should read first.

We are going to make the rather strong assumption that the ecosystem is structured according to exactly the same rules as in the modern ocean, and simply run the same ecological model configuration but in a new climate and ocean environment. However, because we don't really have any good (or any!) data constraints on the iron supply to the early Eocene ocean via e.g. dust, the ecological configuration does not include iron as a limiting nutrient. Bear that in mind when thinking about your results.

The \textit{user-configs} you need to use, or copy-edit-rename, can be found in the directory:
\vspace{1mm}
\\\textsf{\footnotesize genie-userconfigs/MS/wilsonetal.2018}
\vspace{1mm}
\\\noindent(and you can run your experiments from here, or better, copy-edit-rename the \textit{user-configs} you run experiments with, from the \textsf{\footnotesize LABS} sub-directory).

Read the \textsf{\footnotesize readme.txt} file for instructions for the basic set of command line parameters needed, but note that you may be running a \textit{user-config} with a different name and likely from a different sub-directory. You want to spin up the following experiments first (and then experiment with them later):

\vspace{1mm}
\begin{itemize}[noitemsep]
\item[(1)] \textsf{\footnotesize Modern}
\item[(3)] \textsf{\footnotesize Early Eocene CO2 and Climate}
\end{itemize}
\vspace{1mm}

\noindent Submit these to the cluster, but remember that you will need to recompile muffin between (1) and (3) (and re-compile before (1) if you have not been using the \textit{base-config} \texttt{muffin.CBE.worjh2.BASES} immediately prior to this). Run the spin-ups for 10,000 years.

\vspace{2mm}
\noindent When you have completed the pair of spin-ups, see what you can diagnose and learn about how ecosystems and the spatial pattern of ecology and biological export differ between a colder modern ocean and the warmer Eocene ocean.

For example, in the warmer Eocene world:

\vspace{1mm}
\begin{itemize}
\item What has happened to the mean plankton size in different regions?
\item What has happened to the fundamental niches in different size classes?
\item What has happened to the realised niches?
\item Is the system more or less productive?
\item Has carbon export gone up or down?
\end{itemize}
\vspace{1mm}

See what you can find out about the two systems and think about the mechanisms that might be responsible for the differences \dots

\vspace{8mm}
\pagebreak

Note that to be more comparable with the Eocene, this particular modern configuration also lacks iron co-limitation. We could also try and remove the effect of the Eocene being warmer than the modern so as to concentrate just on the effect of a different continental configuration and hence ocean circulation. The experiment (included in the \textit{user-config} sub-directory and briefly described in the \textsf{\footnotesize readme } file:

\vspace{1mm}
\begin{itemize}[noitemsep]
\item[(2)] \textsf{\footnotesize Late Paleocene Early Eocene Paleogeography}
\end{itemize}
\vspace{1mm}

\noindent does exactly this, i.e. attempts to 'remove' the effect of higher ocean temperatures by running an Eocene experiment at \(\times1 CO_{2}\) .

Conversely, we could run modern at \(\times3 CO_{2}\) and then compare to the \(\times3 CO_{2}\) Eocene experiment. This alternative comparative experiment is provided as:

\vspace{1mm}
\begin{itemize}[noitemsep]
\item[(S1)] \textsf{\footnotesize Modern with 3 x CO2}
\end{itemize}
\vspace{1mm}

\noindent and will also require a 10,000 year long spin-up ... Note that the 2 strategies, while slightly different, are attempting the same thing (i.e. isolating the effect of paleography and ocean circulation from coeval climate change and warming).

\vspace{1mm}
Finally -- having run some or all of these spin-ups and contrasted the results (focussing on ecological patterns and biological export, but remembering also to assess differences in ocean circulation), you can investigate the impact of geologically rapid warming a-la the PETM, on the system (esp. ecological patterns and biological export plus ocean circulation). There are two immediately obvious possible approaches:

\begin{enumerate}[noitemsep]

\vspace{1mm}
\item You could use a \(\times1 CO_{2}\) spin-up as a re-start -- either or both of modern and Eocene continental configurations -- and run the \(\times3 CO_{2}\) experiment from this.
\\This will give you an instantaneous warming -- much faster than occurred associated with PETM onset, and also faster even than modern anthropogenic warming. However, it does provide a nice simple idealized perturbation to investigate and analyse.
\\An experiment duration of 100, or even 10 years, might be sufficient.\footnote{But running for a full 10,000 years would enable you to follow not only the initial rapid warming perturbation, but also the long-term recovery and adjustment to a new steady state.}
\\Conversely and for fun, you could also start from a \(\times3 CO_{2}\) \textit{spin-up}, and run a \(\times1 CO_{2}\) experiment, to achieve a rapid cooling. Investigate the differential ecological response to rapid cooling vs. rapid warming.

\vspace{1mm}
\item Secondly, in the \textit{user-configs}, you might note near the bottom is a \textit{forcing} that determines the value of atmospheric \(CO_{2}\):
\vspace{-1mm}\small\begin{verbatim}
# specify forcings
bg_par_forcing_name="pyyyyz.RpCO2_Rp13CO2"
\end{verbatim}\normalsize\vspace{-1mm}
followed by a line that specifies either \(\times1 CO_{2}\) or \(\times3 CO_{2}\), e.g.
\vspace{-1mm}\small\begin{verbatim}
bg_par_atm_force_scale_val_3=278.0E-06
\end{verbatim}\normalsize\vspace{-1mm}
or
\vspace{-1mm}\small\begin{verbatim}
bg_par_atm_force_scale_val_3=834.0E-06
\end{verbatim}\normalsize\vspace{-1mm}
(followed by a line specifying the carbon isotopic composition of the atmosphere).

%------------------------------------------------
\newpage
%------------------------------------------------
%
Similar to as you have seen before for a fossil fuel \(CO_{2}\) emissions \\\textsf{\footnotesize biogem\_force\_restore\_atm\_pCO2\_sig.dat}
which contains:
\vspace{-1mm}\small\begin{verbatim}
-START-OF-DATA-
0.0      1.0
999999.0 1.0
-END-OF-DATA-
\end{verbatim}\normalsize\vspace{-1mm}

Referring back to the instructions for changing fossil fuel \(CO_{2}\) emissions \textit{forcing}, you should be able to modify this (or better, copy-rename a new forcing directory and edit the file \textsf{\footnotesize biogem\_force\_restore\_atm\_pCO2\_sig.dat} in that) to create a prescribed time-dependent change in atmospheric \(CO_{2}\).

Note that in the format of \textsf{\footnotesize biogem\_force\_restore\_atm\_pCO2\_sig.dat}, the values in the second column scale the value of the parameter \texttt{bg\_par\_atm\_force\_scale\_val\_3} in the \textit{user-config}. Hence, in \textsf{\footnotesize biogem\_force\_restore\_atm\_pCO2\_sig.dat}, setting a value of \texttt{2.0} in place of \texttt{1.0}, will double the applied \(CO_{2}\) forcing. Exactly as per in the fossil fuel \(CO_{2}\) emissions exercises, pulses, ramps, etc etc can be constructed to control the rate and shape of the applied change in atmospheric \(CO_{2}\).

\end{enumerate}

\vspace{2mm}

\noindent Either way --  assess the important and impact of the rate of \(CO_{2}\) rise and hence warming. (Equally, you might explore rapid cooling and how that fundamentally differs in impact from warming.)

%------------------------------------------------
\newpage
%------------------------------------------------
%
\subsection*{Marine ecology following the end Cretaceous Impact}



%------------------------------------------------

\newpage

%------------------------------------------------

\section{Ecology in ... fake oceans!}

%------------------------------------------------
%
We can also consider the question of the causes and consequences of different ecologies in the ocean in a more general way and return to our hypothetical or 'fake' oceans.

For any of your 'fake' worlds' that you generated earlier, you can add carbon and nutrient cycling (if that was not already included), plus a marine ecology. To do this, you first need to create a new \textit{base-config} that includes all the carbon cycling and nutrients needed by \textbf{ECOGEM} (and \textbf{BIOGEM}), then you need to create/configure a suitable \textit{user-config}.

\begin{enumerate}[noitemsep]

\vspace{1mm}
\item \textit{base-config} -- First, you need to enable the \textbf{ECOGEM} module.
\\At the top of the your \textit{base-config} file, change the \textbf{ECOGEM} 'flag' to true:
\vspace{-1mm}\begin{verbatim}
ma_flag_ecogem=.TRUE.
\end{verbatim}\vspace{-1mm}
Next, it is likely that you did not add any biogeochemical tracers when you created your fake world, and the \textit{base-config} section:
\footnotesize\begin{verbatim}
# *******************************************************************
# TRACER CONFIGURATION
# *******************************************************************
\end{verbatim}\normalsize
will probably look like:
\footnotesize\begin{verbatim}
# the total number of tracers includes T and S
# T and S do not need to be explicitly selected and initialized
# *******************************************************************
# Set number of tracers
GOLDSTEINNTRACSOPTS='$(DEFINE)GOLDSTEINNTRACS=2'
# list selected biogeochemical tracers
# <<<                                                             >>>
# list biogeochemical tracer initial values
# <<<                                                             >>>
\end{verbatim}\normalsize
Go find the \textit{base-config} file \textsf{\footnotesize muffin.CBE.p0055c.BASES} in the \textsf{\footnotesize configs} sub-directory of \textsf{\footnotesize genie-main}. Copy all of the section headed
\footnotesize\begin{verbatim}
# *******************************************************************
# TRACER CONFIGURATION
# *******************************************************************
\end{verbatim}\normalsize
into your \textit{base-config}, replacing the original contents with only 2 tracers selected.

\vspace{1mm}
\item \textit{user-config} -- Next, you need to define some biogeochemical cycling PLUS and ecosystem. Go find the \textit{user-config} file: \textsf{\footnotesize wilsonetal.p0055c.8P8Z.pal.3x} in \textsf{\footnotesize genie-userconfigs} sub-directory \\ \textsf{\footnotesize MS/wilsonetal.2018}.

Easiest, is to simply re-use (copy-rename) the  \textit{user-config} file: \textsf{\footnotesize wilsonetal.p0055c.8P8Z.pal.3x}.

\vspace{1mm}
The only parameters you might want to adjust\footnote{Note that ocean alkalinity is also set for an Eocene world.}, other than a different ecological structure (and \texttt{eg\_par\_ecogem\_plankton\_file}), is atmospheric \(CO_{2}\), which is set to \(\times 3 CO_{2}\) by:
\small\begin{verbatim}
bg_par_forcing_name="pyyyyz.RpCO2_Rp13CO2"
bg_par_atm_force_scale_val_3=834.0E-06
bg_par_atm_force_scale_val_4=-4.9
\end{verbatim}\normalsize
(and atmospheric \(\delta^{13}C\) to \(-4.9\)).

\vspace{1mm}
Strictly, the scaling for the air-sea gas exchange coefficient, for a fake world, should be:
\footnotesize\begin{verbatim}
# re-scale gas transfer coefficient ...
bg_par_gastransfer_a=0.722
\end{verbatim}\normalsize
(changing the value from \texttt{0.5196} to \texttt{0.722}).

\end{enumerate}

\pagebreak

Those changes -- enabling \textbf{ECOGEM} and adding ocean (and atmosphere) tracers to your \textit{base-config}, and then taking a paleo \textbf{ECOGEM} \textit{user-config} as a template to work from, should get you going with an ecology in your fake world.

\vspace{1mm}
If you also want to diagnose ocean circulation better and add a ventilation tracer, then in the \textit{base-config}, increase the number of selected tracers by 2 (under \texttt{\# Set number of tracers}) and add the following 2 lines to the list of selected tracers:
\footnotesize\begin{verbatim}
gm_ocn_select_48=.true. # r
gm_ocn_select_49=.true. # b
\end{verbatim}\normalsize
and ... in the \textit{user-config}, add (anywhere, but e,g, in the \texttt{MISC} parameter section):
\footnotesize\begin{verbatim}
# add ventillation tracers
bg_ctrl_force_ocn_age=.true.
\end{verbatim}\normalsize

%------------------------------------------------

\newpage

%------------------------------------------------

\section{EcoGEnIE 1.1}

As outlined in the Sections: \textit{Choosing a template user-config} and \textit{Configuring muffin experiments}, a number of recommended changes to the configuration of \textbf{ECOGEM} as published by \textit{Ward et al.} [2018] (EcoGEnIE 1.0).

\vspace{1mm}
These changes have been incorporated into an example \textit{user-config} (in the \textsf{\footnotesize genie-userconfigs } sub-directory \textsf{\footnotesize EXAMPLES}: \textsf{\footnotesize muffin.CB.worlg4.BASESFeTDTL.ECOGEM\_NEW.SPIN} (and paired with the \textit{base-config}: \textsf{\footnotesize muffin.CB.worlg4.BASESFeTDTL}), and are:

\begin{itemize}[noitemsep]

\vspace{2mm}
\item Under \texttt{*** REMINERALIZATION ***}:
\small\vspace{-1mm}\begin{verbatim}
# set 'instantaneous' water column remineralziation
bg_par_bio_remin_sinkingrate_physical=9.9E9
bg_par_bio_remin_sinkingrate_reaction=125.0
\end{verbatim}\vspace{-1mm}\normalsize
which instantaneously remineralizes all particulate organic matter throughout the water column according to the remineralization profile and/or reaction rates. This is activated via a 'very large' value for  \texttt{bg\_par\_bio\_remin\_sinkingrate\_physical}. At the same time, reaction rates (including scavenging) are calculated as if the sinking rate was finite and equivalent to the value of: \texttt{bg\_par\_bio\_remin\_sinkingrate\_reaction} (\(m\:d^{-1}\)). 

\vspace{2mm}
\item Under \texttt{*** MISC ***}:
\small\vspace{-1mm}\begin{verbatim}
# set mixed layer to be only diagnosed (for ECOGEM)
go_ctrl_diagmld=.true.
# add seaice attenuation of PAR
eg_ctrl_PARseaicelimit=.true.
# relative partitioning of C into DOM
eg_par_beta_POCtoDOC=0.75
\end{verbatim}\vspace{-1mm}\normalsize
which firstly substitutes a diagnosed mixed layer depth, rather than actually applying mixed layer physics to ocean circulation, then limits light available under sea-ice in proportion to the fractional sea-ice cover in that grid cell, and lastly, re-partitions carbon from POM to DOM and is tuned to produce an approximately Redfield ratio (\(106\)) of \(104.7:1\) in \(C:P\) of exported POM., and then:
\small\vspace{-1mm}\begin{verbatim}
# maximum time-scale to geochemical reaction completion (days)
bg_par_bio_geochem_tau=90.0
# extend solubility and geochem constant T range (leave S range as default)
gm_par_geochem_Tmin  = -2.0
gm_par_geochem_Tmax  = 45.0
gm_par_carbchem_Tmin = -2.0
gm_par_carbchem_Tmax = 45.0
\end{verbatim}\vspace{-1mm}\normalsize
which firstly, limits the maximum consumption of any particular reactant by a single reaction, to an imposed lifetime (\(90\:days\)), with the remaining parameters extending the valid temperature range for solubility and geochem constants to \(-2 - 45^{\circ}C\). Note that the valid salinity range is left unchanged.

\end{itemize}

\vspace{1mm}
\noindent For reference, the original (i.e. as per in \textit{Ward et al.} [2018]) but reformatted, user-config, is provided as: \textsf{\footnotesize muffin.CB.worlg4.BASESFeTDTL.ECOGEM\_NEW.SPIN} (also in the \textsf{\footnotesize genie-userconfigs} sub-directory \textsf{\footnotesize EXAMPLES}).

%------------------------------------------------
\newpage
%------------------------------------------------

\section{Further ideas and investigations}

\subsubsection{The role of iron limitation}
\vspace{1mm}

Up to this point, we have included both phosphate and iron as limiting nutrients. You might want to know more about the importance (or not) of iron availability in determining patterns of biological productivity.

\begin{enumerate}[noitemsep]

\vspace{1mm}
\item You can get a more exact picture of the nutrient limitation terms via the netCDF output variables: \textsf{\footnotesize eco2D\_xGamma\_Fe\_001} and \textsf{\footnotesize eco2D\_xGamma\_P\_001}.

\vspace{1mm}
These two variables take values of between 0 and 1. A \textsf{\footnotesize 1} indicates that the factor is not limiting to growth. A \textsf{\footnotesize 0} indicates the factor is completely preventing growth.
\vspace{1mm}
\begin{itemize}
\item In what regions are iron and phosphorus more or less limiting to growth? \item In regions where neither is limiting, what other factors might be important?
\end{itemize}

\vspace{1mm}
\item  Plankton stoichiometry plays a critical role in determining which nutrient is most limiting to growth. You can increase the plankton \(Fe:C\) ratio by increasing the minimum and maximum iron quotas. Look at the parameters \texttt{eg\_qminFe\_a} and \texttt{eg\_qmaxFe\_a} in the \textit{user-config} file.

\vspace{1mm}
\begin{itemize}
\item What happens to the ecosystem if you increase these parameters by a factor of 2, 5 or 10?
\item How does a change in these parameters affect the model behaviour?
\item What has changed in terms of the patterns of nutrient limitation?
\item What has happened to the concentration of the limiting and non-limiting nutrient?
\end{itemize}

\vspace{1mm}
\item[NOTE:] Rather than changing the parameter and subsequently forgetting what you started with ... instead, you might copy/paste a new version of the line in question, and comment out the original by placing a `\texttt{\#}' at the beginning of the line. For example:
\vspace{-1mm}\small\begin{verbatim}
#eg_qminFe_a = 3.0e-6
eg_qminFe_a  = 6.0e-6
\end{verbatim}\normalsize\vspace{-1mm}
changes the minimum iron quota by a factor of 2, whilst keeping a record of the original setting (inactivated by the \texttt{\#}). Or, you might do something like:
\vspace{-1mm}\small\begin{verbatim}
eg_qminFe_a  = 6.0e-6 # 3.0e-6
\end{verbatim}\normalsize\vspace{-1mm}
that reduces the total number of lines you end up with in the \textit{user-config} file.

\vspace{1mm}
\item Nutrient supply ratios are also important in determining the limiting nutrient.
\\The \texttt{bg\_par\_det\_Fe\_sol\_exp} parameter determines the solubility of atmospheric iron inputs in seawater. Decreasing the value of \texttt{bg\_par\_det\_Fe\_sol\_exp} will therefore decrease the iron-to-phosphorus supply ratio.

\vspace{1mm}
\begin{itemize}
\item What happens to the ecosystem if you decrease \texttt{bg\_par\_det\_Fe\_sol\_exp} by e.g. 10, 20 or 50\%?
\end{itemize}

\vspace{1mm}
\item Lastly, you can turn off entirely the iron requirement of plankton:
\vspace{-1mm}\small\begin{verbatim}
# include cellular quota: Fe -- no Fe limitation enabled
eg_useFe                    =.FALSE.
eg_fquota                   =.FALSE.
\end{verbatim}\normalsize\vspace{-1mm}
\vspace{1mm}
\begin{itemize}
\item What happens to the patterns of biological productivity as well as global total export?
\end{itemize}

\end{enumerate}

%------------------------------------------------
\vspace{1mm}
\noindent\rule{4cm}{0.1mm}

%----------------------------------------------------------------------------------------
%----------------------------------------------------------------------------------------