%----------------------------------------------------------------------------------------
%       CHAPTER 6
%----------------------------------------------------------------------------------------

\cleardoublepage

\chapterimage{chx-biogeo.png} % Chapter heading image

\chapter{Ocean biogeochemical cycles}\label{ch:ocean-biogeochem}

\hfill \break

\vspace{24mm}

\noindent

%------------------------------------------------
\newpage
%------------------------------------------------

\section*{READ.ME}

You will need the following \textit{re-starts} files prior to embarking on the experiments in any of the listed  Sections in this Chapter.
\vspace{1mm}

\begin{itemize}[noitemsep]
\vspace{2mm}
\item [6.1] This is the basic single (\(PO_{4}\)) nutrient biological export scheme configuration for playing with.
\vspace{-2mm}\footnotesize\begin{verbatim}
$ wget --no-check-certificate 
     http://www.seao2.info/cgenie_output/cookie.CB.p_worbe2.BASES.ridgwelletal.SPIN.tar.gz
\end{verbatim}\normalsize\vspace{-1mm}
\vspace{2mm}
\item [6.2] A single (\(PO_{4}\)) nutrient biological export scheme but in a higher (vertical) resolution ocean.
\vspace{-1mm}\footnotesize\begin{verbatim}
$ wget --no-check-certificate 
     http://www.seao2.info/cgenie_output/cookie.CB.p_worjh2.BASES.caoetal.SPIN.tar.gz
\end{verbatim}\normalsize\vspace{-1mm}
World Ocean\ Atlas modern (\(PO_{4}\)) climatology (re-gridded/interpolated observations), re-gridded to the same ocean grid as \textsf{\footnotesize EXAMPLE.worjh2.Caoetal2009.SPIN}
\vspace{-1mm}\footnotesize\begin{verbatim}
$ wget --no-check-certificate 
     http://www.seao2.info/cgenie_output/worjh2.p_an.200709.nc
\end{verbatim}\normalsize\vspace{-1mm}
World Ocean\ Atlas modern (\(O_{2}\)) climatology (re-gridded/interpolated observations), re-gridded to the same ocean grid as \textsf{\footnotesize EXAMPLE.worjh2.Caoetal2009.SPIN}
\vspace{-1mm}\footnotesize\begin{verbatim}
$ wget --no-check-certificate 
     http://www.seao2.info/cgenie_output/worjh2.o_an.200709.nc
\end{verbatim}\normalsize\vspace{-1mm}
\vspace{2mm}
\item [6.3] This is a paired T-dependent and non-T-dependent export and remineralizaton configuration.
\vspace{-1mm}\footnotesize \begin{verbatim}
$ wget --no-check-certificate 
     http://www.seao2.info/cgenie_output/cookie.CB.p_worjh2.BASES.crichtonetal.STND.SPIN.tar.gz
$ wget --no-check-certificate 
     http://www.seao2.info/cgenie_output/cookie.CB.p_worjh2.BASES.crichtonetal.TDEP.SPIN.tar.gz
\end{verbatim} \normalsize
\vspace{2mm}
\item [6.4] This is a scheme with T-dependent biological export (but with a non T-dependent, fixed remineralization profile) and with Fe co-limitation alongside P.
\vspace{-1mm}\footnotesize\begin{verbatim}
$ wget --no-check-certificate 
     http://www.seao2.info/cgenie_output/cookie.CB.p_worjh2.BASESFe.FeMIP.SPIN.tar.gz
\end{verbatim}\normalsize
\vspace{2mm}
\item [6.5] (same \textit{restart} as per for 6.4)
\end{itemize}

\vspace{2mm}
\noindent Extract the results in the usual way and in the usual place ... and return to \textsf{\footnotesize genie-main} in the usual way ... all ... as usual.

%------------------------------------------------
\newpage
%------------------------------------------------

\section*{The ocean's biological pump}

In this Chapter we'll step through some of the facets of the cycle of carbon (and nutrients) in the ocean -- the 'biological pump'. And then in a later section we'll look at the same processes but in a different way -- through the lens of 'geoengineering', and hopefully learn something further about how everything 'works' regardless of the desirability and effectiveness (or not) of geoengineering.

\textbf{cookie} incorporates a variety of options for simulating biological export production. To date, these have all been rooted in what is effectively a nutrient mass balance approach to estimating export production, nicely encapsulated by Ernst Maier-Reimer as: “\textit{conceptually not a model of biology in the ocean but rather a model of biogenically induced chemical fluxes [from the surface ocean]}” [\textit{Maier-Reimer,} 1993]. Hence in schemes of this nature -- which we will term ‘biogenic  flux’ schemes --  there is no attempt to explicitly account for changes in cell numbers/biomass and hence, nor zooplankton, which will impact a bias particularly in the seasonal time-dependent response of export. Previous biogenic flux schemes utilized in  \textbf{cookie}  considered a single nutrient (P limitation) only and took either restoring to observations [Cameron et al., 2005] or explicit P-limitation [Ridgwell et al., 2007a,b] approaches. More recently, additional limitation by iron (P+Fe) has been implemented, while nitrogen cycling (including N fixation and denitrification) and hence P+N limitation, has been implemented and explored in the context of extreme nutrient and oxygen cycle perturbation associated with the Cretaceous Oceanic Anoxic Events [Monteiro et al., 2012; Naafs et al., 2029]. Work currently in development includes the additional consideration of Si limitation (P+Si limitation) control on production of diatoms vs. non-diatoms following \textit{Ridgwell et al.} [2002].

Overall, biological production and net export of particulate (POM) and dissolved (DOM) organic matter plus calcium carbonate (\(CaCO_{3}\)) are directly determined by the availability of nutrients (phosphate and/or total dissolved iron and/or \(NO^{2-}_{3}\) (and/or \(NH^{+}_{4}\)) and/or dissolved silica) together with the degree to which physical conditions, particularly light and temperature, are conducive to growth. DOM is remineralzied (transformed back to inorganic dissolved constituents) relatively rapidly (a ca. annual time-scale) and hence mostly at or close to the ocean surface, while POM is assumed to sink down into the ocean interior. See: \textit{Hülse at al.} [2017]\footnote{Hülse, D., S. Arndt, J.D. Wilson, G. Munhoven, and A. Ridgwell, Understanding the causes and consequences of past marine carbon cycling variability through models, Earth-Science Reviews 171, dx.doi.org/10.1016/j.earscirev.2017.06.004 (2017).} for a comprehensive review (esp. of implementation in models).

Note that a full description of the various biological and ocean interior remineralization schemes (together constituting the biological pump) is not given here as background. Rather, the relevant provided references should be read (rather than e.g. downloaded and immediately forgotten ...).

%------------------------------------------------
\vspace{1mm}
\noindent\rule{4cm}{0.5pt}
\vspace{2mm}
%------------------------------------------------

%------------------------------------------------
\newpage
%------------------------------------------------

\noindent What should you be thinking of looking at in terms of model output to understanding something about the role of the oceans biological pump in the Earth system (and climate dynamics)? Each subsequent exercise will direct you to some of the outputs to analyze/visualize that are directly relevant to the question (model experiment). In addition to that, what follows is a brief general over-view of relevant fields.

\begin{itemize}[noitemsep]

\vspace{2mm}
\item Firstly, it is worth noting, if you have not already, the summary text file: 
\\\textsf{\footnotesize biogem\_year\_yyyy\_yyy\_diag\_GLOBAL\_AVERAGE.res}
where \textsf{\footnotesize yyyy\_yyy} is the mid-point of the annual average year saved. Hence for the provided 10,000 year spin-up \\\textsf{\footnotesize cookie.CB.p\_worbe2.BASES.ridgwelletal.SPIN}, the file is: \\\textsf{\footnotesize biogem\_year\_09999\_500\_diag\_GLOBAL\_AVERAGE.res}. 

\vspace{1mm}
Each of these files contains summary information associated with each \textit{time-slice}. Model properties that you might pay particular attention to includes:

\vspace{1mm}
\begin{itemize}[noitemsep]
\item
\vspace{1mm}
\vspace{-0mm}\footnotesize\begin{verbatim}
 ATMOSPHERIC PROPERTIES
 Atmospheric pCO2             :    278.000 uatm
 Atmospheric pCO2_13C         :     -6.500 o/oo
\end{verbatim}\normalsize\vspace{1mm}
is a useful place to find e.g. the final (annual average) value of atmospheric \(pCO_{2}\) (rather than searching to the end of the relevant \textit{time-series} file).
\item
\vspace{1mm}
\vspace{-0mm}\footnotesize\begin{verbatim}
 BULK OCEAN PROPERTIES
 Ocean DIC              ..... :   2211.435 umol kg-1 <->   0.2977295E+19 mol
\end{verbatim}\normalsize\vspace{1mm}
reflecting how much carbon is stored in the ocean, 
\vspace{0mm}\footnotesize\begin{verbatim}
 Ocean PO4              ..... :      2.154 umol kg-1 <->   0.2899897E+16 mol
\end{verbatim}\normalsize\vspace{0mm}
the ocean nutrient (here: phosphorous) inventory,
\vspace{-0mm}\footnotesize\begin{verbatim}
 Ocean O2               ..... :    221.288 umol kg-1 <->   0.2979245E+18 mol
\end{verbatim}\normalsize\vspace{0mm}
is the average concentration of dissolved oxygen in the ocean (and its inventory).
\item
\vspace{1mm}
\vspace{-0mm}\footnotesize\begin{verbatim}
 SURFACE EXPORT PRODUCTION
 Export flux POC              :    204.253 umol cm-2 yr-1 <->   0.7505843E+15 mol yr-1
 Export flux CaCO3            :     28.624 umol cm-2 yr-1 <->   0.1051875E+15 mol yr-1
\end{verbatim}\normalsize\vspace{1mm}
fluxes reflecting the global biological export of particulate organic matter and calcium carbonate, and then
\vspace{-0mm}\footnotesize\begin{verbatim}
 SURFACE EXPORT PRODUCTION
 Export flux POC              :    204.253 umol cm-2 yr-1 <->   0.7505843E+15 mol yr-1
 Export flux CaCO3            :     28.624 umol cm-2 yr-1 <->   0.1051875E+15 mol yr-1
\end{verbatim}\normalsize\vspace{0mm}
is the residual flux that actually reaches the sea-floor.
\end{itemize}

\vspace{1mm}
Finally, there is a summary of the summary, which is often the most useful of all:
\vspace{1mm}
\begin{itemize}[noitemsep]
\item
\vspace{-0mm}\footnotesize\begin{verbatim}
 SURFACE EXPORT & SEDIMENT DEPOSITION (RAIN) FLUX SUMMARY
 Total POC export   :   0.7505843E+15 mol yr-1 =   9.007 PgC yr-1
 Total CaCO3 export :   0.1051875E+15 mol yr-1 =   1.262 PgC yr-1
 Total POC rain     :   0.8654062E+14 mol yr-1 =   1.039 PgC yr-1
 Total CaCO3 rain   :   0.5191799E+14 mol yr-1 =   0.623 PgC yr-1
\end{verbatim}\normalsize\vspace{0mm}
where carbon fluxes are also given in helpful and literature-friendly units of \(PgCyr^{-1}\).
\end{itemize}

%------------------------------------------------
\newpage
%------------------------------------------------
%
\vspace{2mm}
\item Secondly, there are a number of biological pump related \textit{time-series} outputs, e.g.: 

\vspace{1mm}
\begin{itemize}[noitemsep]
\item []
\begin{itemize}[noitemsep]
\item [] \textsf{\footnotesize biogem\_series\_fexport\_POC.res}
\item [] \textsf{\footnotesize biogem\_series\_fexport\_POP.res}
\item [] \textsf{\footnotesize biogem\_series\_fexport\_CaCO3.res}
\end{itemize}
\vspace{1mm}
are the \textit{time-series} of the carbon and phosphorous of particulate organic matter export, and that for \(CaCO_{3}\), and
\vspace{1mm}
\begin{itemize}[noitemsep]
\item [] \textsf{\footnotesize biogem\_series\_ocnsed\_POC.res}
\item [] \textsf{\footnotesize biogem\_series\_ocnsed\_POP.res}
\item [] \textsf{\footnotesize biogem\_series\_ocnsed\_CaCO3.res}
\end{itemize}
\vspace{1mm}
are the corresponding fluxes at the sea-floor. The \textit{time-series} output:
\begin{itemize}[noitemsep]
\item [] \textsf{\footnotesize biogem\_series\_ocn\_PO4.res}
\end{itemize}
will tell you how what the global annual average nutrient concentration is (left) at the surface, and
\begin{itemize}[noitemsep]
\item [] \textsf{\footnotesize biogem\_series\_ocn\_O2.res}
\end{itemize}
the average concentration dissolved oxygen in the ocean and at the sea-floor.
\end{itemize}

\vspace{2mm}
\item Thirdly, as always -- \textbf{netCDF} outputs.

\begin{itemize}[noitemsep]
\vspace{1mm}
\item In the 2D \textbf{netCDF}:
\vspace{1mm}
\begin{itemize}[noitemsep]
\item [] \textsf{\footnotesize bio\_export\_POC}
\item [] \textsf{\footnotesize bio\_export\_POP}
\item [] \textsf{\footnotesize bio\_export\_CaCO3}
\end{itemize}
are spatial fields of the export flux of \(POC\), \(POP\), \(CaCO_{3}\), and variables with \textsf{\footnotesize focnsed} in place of \textsf{\footnotesize bio\_export} in its name the flux distributions to the sea-floor.

\vspace{1mm}
Derived from these:
\vspace{1mm}
\begin{itemize}[noitemsep]
\item [] \textsf{\footnotesize misc\_sur\_rCaCO3toPOC}
\item [] \textsf{\footnotesize misc\_sur\_rPOCtoPOP}
\end{itemize}
are the rations of \(CaCO_{3}/POC\) and \(POC/POP\), respectively.
\vspace{1mm}
\\There are then some fields for surface and benthic tracer concentrations, such as of phosphate and dissolved oxygen.
\vspace{1mm}
\item In the 3D \textbf{netCDF}, \textsf{\footnotesize bio\_*} are the 3D spatial distributions of particles setting down through the water column -- \textsf{\footnotesize bio\_fpart\_*} is the particulate flux density, \textsf{\footnotesize bio\_fparttot\_*} the total flux associated with each grid point, and \textsf{\footnotesize bio\_fpartnorm\_*} are the fluxes in the water column normalized to the export flux out of the base of the surface ocean layer.
\vspace{1mm}
\\Also see spatial distributions of dissolved carbon, nutrients, and oxygen etc -- \textsf{\footnotesize ocn\_*}.
\end{itemize} 

\end{itemize}

%------------------------------------------------
\vspace{1mm}\noindent\rule{4cm}{0.5pt}\vspace{2mm}
%------------------------------------------------

\noindent From just the \textit{spin-up} experiment results you can explore and/or plot some of these fields. Or better -- construct a control experiment following on from the provided \textit{restart} and analyze/explore that.

%------------------------------------------------
\newpage
%------------------------------------------------

\section{Basic controls on biological productivity}

The following exercises  will utilize a very basic (but relatively uncomplicated and fast) representation of the biological pump -- one with only a single nutrient (PO$_{4}$) potentially limiting to biological export, considered. The scheme is described in full in \textit{Ridgwell et al.} [2007]\footnote{Ridgwell, A., et al., Marine geochemical data assimilation in an efficient Earth System Model of global biogeochemical cycling, Biogeosci. 4, 87-104 (2007). }. Note that this specific configuration of the (modern) ocean model accounts only for 8 layers in the ocean (but spanning the same \(0-5000 \) water depth range as the 16-level model configuration used in evaluating AMOC stability). Although the limited vertical resolution of the 8-level model places additional constraints on the ability of the model to reproduce features such as oxygen minimum zones in the ocean, it does enable a much shorter run-time (via a longer time-step in the model as well as calculations only needing to be carried out for about \(50\%\) of the total number of 16-level ocean cells). Hence we are making  trade-off between model fidelity (in reproducing observed distribution of tracers in the modern ocean) and speed (and hence the ability to more effectively 'play' with the Earth system in a relatively short amount of real time).

\vspace{1mm}

We will consider the following scientific questions as a starting point, and then devise some model experiments to address them:

\begin{enumerate}[noitemsep]
\vspace{1mm}
\item What would the global carbon cycle look like with no biological production in the ocean? \\Or alternatively: how important is the biological pump in controlling atmospheric \(pCO_{2}\) and \uline{how much higher would \(pCO_{2}\) be today in the absence of an active marine biosphere}?
\vspace{1mm}
\item Conversely, is the biological pump operating at its maximum efficient in the ocean today, and if not, \uline{how much lower would atmospheric \(pCO_{2}\) be if virtually all nutrients at the surface were consumed}?
\vspace{1mm}
\item Associated with \#1 and \#2 -- how does changing the biological pump affect the distribution and hence availability of dissolved oxygen in the ocean? (How more oxygenated would the ocean be with no export of organic matter into the ocean interior, and how much more severe would oxygen depletion be if plankton were able to utilize all the nutrients at the ocean surface?)
\vspace{1mm}
\item \uline{If carbon and nutrients are returned back to solution (remineralized) much closer to the ocean surface, does atmospheric \(pCO_{2}\) increase or decrease}??? (The faster return of DIC to the ocean surface will tend to increase \(pCO_{2}\) while the return of nutrients will tend to enhance biological productivity and decrease \(pCO_{2}\) ... so the answer is not necessarily intuitive and hence why we build and run computer models.)
\vspace{1mm}
\item Conversely, what if carbon and nutrients are much more efficiently exported into the very deepest parts of the ocean -- does \(pCO_{2}\) increase or decrease (and what happens to e.g. \([O_{2}]\))?
\vspace{1mm}
\item What role does the calcium carbonate (\(CaCO_{3}\)) 'counter pump' play? \uline{Would \(pCO_{2}\) be higher or lower (and by how much) in the absence of calcifying organisms and the production of \(CaCO_{3}\) in the ocean}?
\vspace{1mm}
\item How sensitive is the cycling of carbon and nutrients and hence the pattern of ocean oxygenation to changes in the large-scale circulation of the ocean?. For instance, \uline{would \(pCO_{2}\) be higher or lower (and by how much) with a weaker AMOC}?
\vspace{1mm}
\item What about feedbacks with climate -- how does changing global surface temperature (and hence ocean circulation patterns and sea-ice extent) affect biological productivity and the biological pump in the ocean? (Is this a positive or negative feedback on climate change?)
\end{enumerate}
\vspace{1mm}

%------------------------------------------------
\newpage
%------------------------------------------------
%
Experiments modifying the biological pump in the ocean would ideally be run for a few thousand and perhaps as long as 5000 years in order to obtain a complete new (quasi) steady state of carbon and nutrient cycling in the ocean. However, the main impacts in the ocean and on atmospheric \(pCO_{2}\) often tend develop relatively rapidly (decades). Hence model experiments could be run for a few 10s or perhaps a few 100 years to see something 'happen'.  A full 5000 years can also be run, and you can periodically download and viewing the results as the experiment progresses and not necessarily wait until the end of the experiment to determine a good answer. As always -- if possible -- run a representative (or perhaps the most extreme, such as shutting off the biological pump entirely) experiment for 5,000 or even 10,000 years, and answer for yourself how long different key facets of the system (ocean circulation, global export production, mean ocean \([O_{2}]\), etc.) take to adjust. Don't forget to submit your jobs to the cluster!

%------------------------------------------------
\vspace{1mm}\noindent\rule{4cm}{0.5pt}\vspace{2mm}
%------------------------------------------------

\noindent All the experiments in this first section will start from the same \textit{re-start}\footnote{As per described in Ridgwell et al. [2007].} and are based on the same \textit{user-config}\footnote{The same \textit{user-config} as used to generate the \textit{re-start} (with the exception that the atmospheric \(CO_{2}\) concentration is no longer prescribed).}. To run an e.g. 100 year experiment (but it need not be this -- see above) using the template \textit{user-config} and provided \textit{re-start} would look like:
\vspace{-1mm}\small\begin{verbatim}
./runcookie.sh cookie.CB.p_worbe2.BASES LABS LAB.6.1.EXAMPLE 100 
   cookie.CB.p_worbe2.BASES.ridgwelletal.SPIN
\end{verbatim}\normalsize\vspace{-1mm}
(When you run new experiments based on this, remember to copy and rename the provided \textit{user-config}\textsf{\footnotesize LAB.6.1.EXAMPLE} in order to create new and unique experiments each time.)

\vspace{1mm}

Whatever you plan to do re. perturbation experiments to explore the working of the ocean's biological pump, the first thing to do, is to run a \textbf{control experiment} for however long you have decided to run all the actual experiment experiments for. (You need not wait for this to finish, but can submit it as a \textit{job} to the cluster and get on with running some real experiment experiments.) The control experiment can simply be derived (copied) from \textsf{\footnotesize LAB.6.1.EXAMPLE}, with no further alterations in parameter values needed. In this \textit{user-config}, the value of atmospheric  \(pCO_{2}\) is not prescribed (there is no \textit{forcing} defined) and hence \(pCO_{2}\) is free to respond to any change in parameters. In the case of the control -- there are no changes in the biological pump parameters compared to the \textit{spin-up}, so you should see no (or little) drift in \(pCO_{2}\) occurring. 
A substantive drift in the control experiment (e.g. \(10s\) of \(ppm\) of \(CO_{2}\) in the atmosphere over 100 years) is your way of knowing that you have either used the wrong \textit{re-start} or have accidentally modified key parameters in your control experiment \textit{user-config} compared to those used to generate the \textit{re-start}.

\vspace{1mm}
Plan in advance  the output you want to view and make sure it is going to appear(!) In the \textit{user-config} -- check that the results/output select 'level' is going to give you the output you are expecting. In the example \textit{user-config}, this is specified by:
\vspace{-1mm}\small\begin{verbatim}
bg_par_data_save_level=7
\end{verbatim}\normalsize\vspace{-1mm}
(Refer to Chapter 13 for more details on how different (2D and 3D) \textit{time-slice} variable fields and \textit{time-series} are selected to be included in the model output.)

\vspace{1mm}
Also note that in the example \textit{user-config} provided, netCDF time-slices are only saved at the very end (final annual average) of the model experiment, regardless of how long that is:
\vspace{-1mm}\small\begin{verbatim}
bg_par_infile_slice_name='save_timeslice_NONE.dat'
\end{verbatim}\normalsize\vspace{-1mm}
You may want to request more frequent saving of the spatial fields -- see Section 13.2 (and 13.3).

\vspace{1mm}
In the example \textit{user-config}, \textit{time-series} saving is set to every year.

%------------------------------------------------
\newpage
%------------------------------------------------
%
\noindent In terms of the specific 'questions' at the start (the numbering system is the same in the following), the parameters and parameter values to change and well as what to 'look for' are:
\vspace{1mm}

\begin{enumerate}[noitemsep]

\vspace{1mm}
\item How to kill he biological pump in the ocean?
\\The parameter scaling the rate of nutrient uptake (and hence biological export) is:
\vspace{-1mm}\small\begin{verbatim}
# maximum rate of conversion of dissolved PO4 into 
# organic matter by phytoplankton (mol kg-1 yr-1)
bg_par_bio_k0_PO4=1.9582242E-06
\end{verbatim}\normalsize\vspace{-1mm}
(in units of \(mol\:PO_{4} \:kg^{-1} \:yr^{-1}\)). Setting this to zero would 'turn off' completely the biological pump, leaving you with a abiotic ocean. Alternatively you could initialize the ocean with a zero phosphate concentration.
\vspace{1mm}
\\What to look for? The value of atmospheric \(pCO_{2}\). You might also confirm that biological export really is zero (so check the export fluxes).
\vspace{1mm}
\\To obtain these values you can refer to the respective time-series results files. Or,  the summary text file: \textsf{\footnotesize biogem\_year\_yyyy\_yyy\_diag\_GLOBAL\_AVERAGE.res}
where \textsf{\footnotesize yyyy\_yyy} is the mid-point of the annual average year saved. Each of these files contains summary information associated with  each \textit{time-slice}, including mean atmospheric composition, mean ocean composition, biological export fluxes. The same information can be found in various \textit{time-series} files, but sometimes it is simply easier to find it here in one place!

\vspace{1mm}
\item Maxing out the biological pump?
\\Similar to above -- nutrient uptake and the strength of the biological pump can be enhanced by increasing the value of the parameter \texttt{bg\_par\_bio\_k0\_PO4} (maybe 10 times larger?).
The question is then what does the ocean and atmospheric \(pCO_{2}\) look like if the biological pump is operating at its maximum rate and (almost) all nutrients are consumed at the surface.
\\You may find regions of nutrients that still never get fully consumed ... why?

\vspace{1mm}
\item For the question on dissolved oxygen (\([O_{2}]\)) availability in the ocean interior -- the experiments are the same as \#1 and \#2 (above), except you are looking for how the mean and distribution of dissolved oxygen in the ocean changes. Mean ocean [\(O_{2}\)] values can be obtained form time-series of the summary file. Spatial distributions are recorded in the \textit{netCDF} output. You might view horizontal or vertical slices (from 3D), or benthic distributions (2D). Think about where animals tend to live and where they could potentially be impacted if export from the ocean surface was much higher than 'today' (i.e., compared to the \textit{spin-up}, or ideally, the same year in your \uline{control}).

\vspace{1mm}
\item What would the ocean look like if organic matter was remineralized much closer to the ocean surface?
\\The parameter controlling the depth-scale (and vertical distribution) at which particulate organic matter (POM) is remineralizated in the ocean interior is controlled by:
\vspace{-1pt}\small\begin{verbatim}
bg_par_bio_remin_POC_eL1=550.5195
\end{verbatim}\normalsize\vspace{-1pt}
(units of \(m\)). Reducing the value of this parameter forces a greater proportion of sinking POM to be remineralized closer to the surface, while a larger value pushes particulate organic matter  (and associated carbon and nutrients) deeper down on average into the ocean interior.

\vspace{1mm}
\noindent What to look for? Largely as before -- the summary file is useful for atmospheric \(CO_{2}\), mean ocean {\(O_{2}\)}, and global (biological) export fluxes (as you might expect nutrients released closer to the ocean surface to result in greater surface ocean nutrient supply and hence fuel higher export). You might also consider the spatial patterns of nutrients and dissolved oxygen.
\\See \textit{Meyer et al.} [2016]\footnote{Meyer, K.M., A. Ridgwell, and J.L. Payne, The influence of the biological pump on ocean chemistry: implications for long-term trends in marine redox chemistry, the global carbon cycle, and the evolution of marine animal ecosystems, Geobiology, DOI: 10.1111/gbi.12176 (2016).} for an example of a study testing these sort of model changes and its implications (and hence ideas of what else to look for, and for what a 'reasonable' parameter value to test might be).

\vspace{1mm}
\item Conversely, you can also increase the value of the depth scaling parameter in order to force a deeper mean depth of POM remineralization. 
\\Alternatively -- you can partition more of the total export into a POM form that is assumed to be resistent to degradation and is transported to the ocean floor completely unchanged (see: \textit{Ridgwell et al.} [2007]):
\vspace{-1pt}\small\begin{verbatim}
bg_par_bio_remin_POC_frac2=6.4591110E-02
\end{verbatim}\normalsize\vspace{-1pt}
Increasing this value forces a greater fraction of the POM exported from the surface ocean to reach the ocean floor.\footnote{See Ridgwell et al. 2007] for a description of how this all works.}

\vspace{1mm}
In both (4) and (5), it may not be obvious which way atmospheric \(pCO_{2}\) responds -- e.g. for (4) by returning nutrients more efficiently to the ocean surface, you increase export, and hence increase the fixation and removal of atmospheric \(CO_{2}\) from the ocean surface, drawing down atmospheric \(pCO_{2}\). BUT, at the same time, carbon released from \(POC\) (as \(DIC\)) through bacterial remineralization, also occurs at a shallower depth and is returned to the surface more efficiency ... tending to increase atmospheric \(pCO_{2}\) ... Just this problem -- 2 opposing effects with an uncertain net impact -- is exactly why (to find out the answer) you build and run numerical models of the system!

\vspace{1mm}
\item The export of \(CaCO_{3}\) from the ocean surface is calculated as a ratio to the export of \(POC\). By default, this ratio varies with the carbonate chemistry (saturation state) of the surface ocean, following \textit{Ridgwell et al.} [2007a] (and see also \textit{Ridgwell et al.} [2007b] and \textit{Ridgwell et al.} [2009]). 

\vspace{1mm}
The scaling parameter controlling the molar ratio of \(\frac{CaCO_{3}}{POC}\) is:
\vspace{-1pt}\small\begin{verbatim}
bg_par_bio_red_POC_CaCO3=0.044372
\end{verbatim}\normalsize\vspace{-1pt}
Setting this zero will effectively turn back the clock \(200Ma\) to a world prior to the evolution of planktic calcifiers (e.g. see: \textit{Ridgwell} [2005]), or conversely, a future world in which they are all driven extinction from ocean acidification ... 
\vspace{1mm}
\\What is the impact on atmospheric \(pCO_{2}\) of setting this to zero? From your knowledge of carbonate chemistry ... why does this happen? You might try e.g. doubling the value (and hence doubling the export of \(CaCO_{3}\) from the surface ocean). As well as \(pCO_{2}\), you might also look at fields of \(ALK\) (or other carbonate chemsitry parameters) in the ocean to see what other geochemical changes occur.

Because (rightly or wrongly) in the model, \(CaCO_{3}\) depends on surface ocean \(\Omega\) (w.r.t. calcite) you might explore what happens to \(CaCO_{3}\) production and export under a release of (fossil fuel) \(CO_{2}\) -- simplest here is to use the same re-start, and create and add a forcing to the \textit{user-config} to implement a large pulse or continuous \(CO_{2}\) flux to the atmosphere

%------------------------------------------------
\newpage
%------------------------------------------------
%
\item While it is possible to add a freshwater 'hosing' forcing (as used in an earlier Chapter/exercise) to modify large-scale ocean circulation, there is a simpler way just to 'turn off' the AMOC in the standard modern continental configuration of \textbf{cookie}.

\vspace{1mm}
The consequence of using the simplified \textbf{EMBM} atmosphere in conjunction with no topography over land, is that the net moisture transport from the Atlantic to the Pacific in the real world, is not reproduced. (This net transport is primarily a consequence of the blocking of Westerly transport moisture across the North American continent by e.g. the Rockies, while low latitude Trade Winds travel relatively topographically unrestricted through Central America, meaning that moisture transport from the Pacific to North Atlantic is partly blocked, while the reverse lower latitude transport is now.) To correct for this, there is a built-in prescribed moisture transport (technically, a negative salinity transport), principally from the North Atlantic to the North Pacific. In the model, there is a scaling parameter for the magnitude this transport ... which can be set to zero ... hence removing the prescribed moisture transport and likely leading to a collapse of the AMOC.

\vspace{1mm}
To kill (collapse) (hopefully) the \textbf{AMOC} in the modern continental configuration of \textbf{cookie}, set:

\vspace{-1mm}\small\begin{verbatim}
ea_28=0.0
\end{verbatim}\normalsize\vspace{-1mm}

Stuff to look for -- in addition to how the AMOC changes (e.g. plot the stream-function) how do the patterns of \([PO_{4}\)] and \([O_{2}\)], particularly in the deep North Atlantic change? What impact does this have on global export production, and ultimately on atmospheric \(pCO_{2}\)?

\vspace{1mm}
\item Finally, you might explore how in some (or all) of the above experiments,  a changing climate in response to changing atmospheric \(pCO_{2}\), in turn modulates the impacts through carbon-climate feedback. 

\vspace{1mm}
By default in the \textit{user-config}, climate-\(CO_{2}\) feedback is disabled:
\vspace{-1pt}\small\begin{verbatim}
ea_36=n
\end{verbatim}\normalsize\vspace{-1pt}
i.e. changing atmospheric \(pCO_{2}\) does not influence the climate system (which remains implicitly forced with a preindustrial radiative forcing value). This allows us to avoid complications arising due to carbon-climate feedback and hence obtain a the underlying response of the system to changing the biological pump in the ocean.

\vspace{1mm}
You can re-enable climate feedback by setting:
\vspace{-1mm}\small\begin{verbatim}
ea_36=y
\end{verbatim}\normalsize\vspace{-1mm}
Impacts may include (but not be limited to): ocean stratification (rapid warming) and/or changes in convection at high latitudes, increased/decreased sea-ice extent that affects the ocean surface area available for biological export, changing solubility of oxygen in sea-water.
\\Note that in this particular simple biological export scheme, there is no temperature-dependence of biological activity and hence export.

\end{enumerate}

%------------------------------------------------
\vspace{1mm} \noindent\rule{4cm}{0.5pt} \vspace{2mm}
%------------------------------------------------

\noindent And that pretty much wraps up all the main knobs that it is possible to play with using the very basic biological pump scheme in \textbf{cookie}. (However, a few further ideas follow ...)

%------------------------------------------------
\newpage
%------------------------------------------------

\section{Comparing model vs. observations}

Key  to having any 'confidence' in both past and future biogeochemical (e.g. carbon, nutrient, oxygen) cycling and climate dynamics and sensitivity to perturbation in a model, is having confidence in the model's ability to adequately reproduce the relevant features of the modern ocean in some direct comparison made with observations. Note that it is not necessarily critical that every single feature in the modern ocean is faithfully accounted for -- the degree to which the model needs  to match observations is going to depend on the question. For example, if a model is only needed for making first order projections of changes in atmospheric \(pCO_{2}\) or \(pO_{2}\), the details of the structure of biogeochemical cycling in the ocean may be less important, as long as the gross partitioning of carbon between deep ocean vs. surface and atmosphere, or  bottom water oxygenation and carbon flux to the sediments, respectively, is reasonable. Other questions such a involving denitrification may require more explicit details of the distribution and intensity of oxygen minimum zones (OMZs) to be able to be reproduced in the modern ocean. However, even in the latter case, if structurally in the model OMZs tend to be structurally (e.g. as a property of the fundamental ocean grid and physics and/or biases in circulation) too weak (too high \([O_{2}])\), it is always possible to adjust denitrification (in this example) to occur at the 'correct' rate by changing parameter values (e.g. of a minimum \([O_{2}])\) value at which denitrification occurs). As long as the same structural biases are present in paleo simulations, there is no reason to believe that projected past ocean denitrification rates would be incorrect. (In fact, this adjusting of parameter values to correct for some bias and achieve an appropriate rate of some process, is ubiquitous throughout all Earth system modelling, including atmospheric physics and particularly cloud formation.)

%------------------------------------------------
\vspace{1mm} \noindent\rule{4cm}{0.5pt} \vspace{2mm}
%------------------------------------------------

\noindent As an example/practice in model-data comparison, the output of a higher vertical resolution ocean model simulation (\(16\) vs. the previous \(8\) levels in the ocean) -- \textsf{\footnotesize cookie.CB.p\_worjh2.BASES.caoetal.SPIN} -- is provided here (see \textbf{READ.ME} downloads) along with observations of the distributions of \([PO_{4}\)] and \([O_{2}\)] in the modern ocean. The latter (observations) are interpolated and re-gridded to a high (\(1\) degree in lon and lat) resolution grid and available as part of the World Ocean Atlas series of ocean climatologies (both physics and geochemical), and then re-gridded to the same \textbf{cookie} ocean grid as the provided (16 ocean level) experiment (which is the spin-up of \textit{Cao et al.} [2009]).

\vspace{1mm}

While the comparison can be made more formally and statistically (with a little \textbf{MATLAB} or \textbf{python} code), much can be gain visually (and indeed this is typically how model-data comparisons are made in the literature). Fortunately, this visual comparison can be made in \textbf{Panoply} and both re-gridded observations and model output, being \textbf{netCDF} format, can be loaded into \textbf{Panoply}. When making visual comparisons, make sure that you use the same scale limits in both plots (these can be set manually). You can also create difference plots of model minus observations. For difference plots -- be careful when changing depth or longitude slices, that you change to the same slice in both model and observations. Note that for difference plots the convention is to use a color-scale that goes from blue (negative) to red (positive) through white (or sometimes, a yellow). Set the magnitude of the scale limits equal so that zero (no difference between model and data) maps onto white (or yellow).

You may (and ideally should!) have ideas/thoughts on 'why' the model and observations diverge where they do. These thoughts may well involve processes in the ocean, e.g. is the dissolved oxygenation feature x due to insufficient ro excessive organic matter remineralization in the ocean, or is the surface surface dissolved phosphate concentration in region y excessive/over-depleted due to insufficient/excessive biological productivity? Perhaps the most fundamental purpose or advantage of models is that they provide you a way to test and attempt to answer such questions (which are basically questions of understanding global biogeochemical cycles in the first place and how tracer patterns in the ocean arise). So given a hypothesis for model-data mismatch, you might return to the previous section and see if a model parameter is listed that could be adjusted in value to test your hypothesis (e.g., changing reminerilization depth or changing a control on biological export might improve or worsen the model-data git -- often it can improve in one location and degrade elsewhere).

Note that the first section of this chapter uses a faster 8-level ocean model configuration, whereas the model-data comparisons are on the basis of a 16-level ocean and so the results of testing parameter changes in an 8-level ocean should be regarded as qualitatively rather than quantitatively comparable.

%------------------------------------------------
\vspace{1mm} \noindent\rule{4cm}{0.1mm} \vspace{2mm}
%------------------------------------------------

\noindent If you wish to play with (modify and test hypotheses etc.) this configuration, and/or run a control experiment to make the comparisons with observations with, an example \textit{user-config} (with fixed climate and no $CO_{2}$-climate feedback) is provided as: \textsf{\footnotesize LAB.6.2.EXAMPLE} and can be run"

\vspace{-1mm}\small\begin{verbatim}
./runcookie.sh cookie.CB.p_worjh2.BASES LABS LAB.6.2.EXAMPLE 100 
   cookie.CB.p_worjh2.BASES.caoetal.SPIN
\end{verbatim}\normalsize\vspace{-1mm}

%------------------------------------------------
\newpage
%------------------------------------------------

\section{Temperature and the biological pump}

So far, both 8- (non-seasonal) and 16- (seasonally forced) level configurations have not account for the role of temperature in biological (metabolic) activity and rates of carbon transformation.

\vspace{1mm}
A pair of example configurations are provided for you as per published in \textit{Crichton et al.} [2021]\footnote{Crichton, K. A., J. D. Wilson, A. Ridgwell, P. N. Pearson, Calibration of temperature-dependent ocean microbial processes in the cGENIE.muffin (v0.9.13) Earth system model, GMD 10.5194/gmd-14-125-2021 (2021)}. 

\begin{enumerate}[noitemsep]
\vspace{1mm}
\item The \textit{user-config} for the non temperature-dependent ('standard') control configuration of \textit{Crichton et al.} [2021]:
\\\textsf{\footnotesize LAB.6.3.STND} 
\\and the resulting \textit{spin-up} which you can use as a \textit{re-start}: 
\\\textsf{\footnotesize cookie.CB.p\_worjh2.BASES.crichtonetal.STND.SPIN}
\\Note that this configuration is basically the same as in \textit{Cao et al.} [2009].
\vspace{1mm}
\item The \textit{user-config} for the temperature-dependent (both biological uptake and water column remineralization) configuration of \textit{Crichton et al.} [2021]:
\\\textsf{\footnotesize LAB.6.3.TDEP} 
\\and the corresponding spin-up: 
\\\textsf{\footnotesize cookie.CB.p\_worjh2.BASES.crichtonetal.TDEP.SPIN}
\end{enumerate}

\vspace{1mm}
For both, the \textit{base-config} you need is the same: \textsf{\footnotesize cookie.CB.p\_worjh2.BASES}

%------------------------------------------------
\vspace{1mm} \noindent\rule{4cm}{0.1mm} \vspace{2mm}
%------------------------------------------------

\noindent One thing to do is simply to explore how the biogeochemical cycling (export flux magnitudes and patterns, nutrient and dissolved oxygen distributions at the surface and down through the water column) differs. You might then apply a warming perturbation by adjusting the scaling value for atmospheric \(pCO_{2}\) in the prescribed restoring \textit{forcing}:

\vspace{-2mm}\small\begin{verbatim}
# *** FORCINGS ******************************************************
...
bg_par_atm_force_scale_val_3=280.0E-06
\end{verbatim}\normalsize\vspace{-2mm}
and see what 'happens' (contrast the response of non T-dependent vs. T-dependent configurations under the exact same perturbation). \textbf{And then ... try and deduce why whatever has happened, has happened ...}

Note that in these \textit{user-configs}, atmospheric \(pCO_{2}\) is restored to a given value and this controls climate (rather than was the case with \texttt{\small ea\_36=n}).

%------------------------------------------------
\vspace{1mm} \noindent\rule{4cm}{0.1mm} \vspace{2mm}
%------------------------------------------------

%------------------------------------------------
\newpage
%------------------------------------------------

\noindent You could also swap out the fixed prescribed atmospheric \(pCO_{2}\) restoring forcing:
\vspace{-2mm}\small\begin{verbatim}
# specify forcings -- generic forcing of atmopsheric pCO2 and d13C
# NOTE: the original paper used 280 ppm
bg_par_forcing_name="pyyyyz.RpCO2_Rp13CO2"
bg_par_atm_force_scale_val_3=280.0E-06
bg_par_atm_force_scale_val_4=-6.5
\end{verbatim}\normalsize\vspace{-2mm}
with an emissions forcing as per you user previously exploring the impact of fossil fuel $CO_{2}$ release. 

\vspace{1mm}
For example, from Chapter 5, the basic emissions flux forcing was:
\vspace{-2mm}\small\begin{verbatim}
# specify forcings -- flux of CO2 to atmosphere
# NOTE: bg_par_atm_force_scale_val_3 == scaling of units from mol yr-1 to PgC yr-1
# NOTE: bg_par_atm_force_scale_val_4 == carbon isotopic composition of CO2 emissions
bg_par_forcing_name="pyyyyz.FpCO2_Fp13CO2"
bg_par_atm_force_scale_val_3=8.3333e+013
bg_par_atm_force_scale_val_4=-27.0
\end{verbatim}\normalsize\vspace{-2mm}
which you could paste into the \textit{user-config} (in place of the restoring forcing) and modify as before for a more exciting perturbation of the system.

\vspace{2mm}
\noindent Experiments to configure and run that you might like to consider are:

\begin{enumerate}[noitemsep]
\vspace{1mm}
\item Drive both 'standard' and T-dependent configuration with the same emissions scenario.
\\This contrasts the response, including climate feedback, of 'standard' and T-dependent scheme and gives you some quantitative sense of the important of T-dependent processes in the ocean and the strength and sign of their associated feedbacks.
\vspace{1mm}
\item Run both the above pair of experiments with fixed climate (radiative forcing), which for the assumption made in the paper of $280.0\:ppm\:pCO_{2}$ would require:
\vspace{-1mm}\small\begin{verbatim}
ea_36=n
ea_radfor_scl_co2=1.007
\end{verbatim}\normalsize\vspace{-1mm}
(where $280/278 = 1.007$)
\\This would help you isolate the strength and sign of the feedback of both systems (and $pCO_{2}$) with climate.
\vspace{1mm}
\item Strictly ... you would run 4 controls -- one for both 'standard' and T-dependent and for both fixed and $CO_{2}$-responsive climate. All 4 control experiments would be run with no carbon emissions.
\end{enumerate}

%------------------------------------------------
\vspace{1mm} \noindent\rule{4cm}{0.1mm} \vspace{2mm}
%------------------------------------------------

\noindent What to look out for?

\begin{itemize}[noitemsep]
\vspace{1mm}
\item Changes in $pCO_{2}$ between standard and T-dep configurations and with and without carbon-climate feedback (\texttt{\small ea\_36}) enabled.
\vspace{1mm}
\item Changes in biological export between standard and T-dep configurations in response to.
\vspace{1mm}
\item Latitude-vertical slices of dissolved $PO_{4}$ and $O_{2}$ for the T-dep experiments would reveal how nutrient cycling and the intensity and depth of OMZs might respond to a warming climate. 
\end{itemize}

%------------------------------------------------
\newpage
%------------------------------------------------

\section{Iron co-limitation of biological productivity}

"Iron (\(Fe\)) is a “micronutrient”; essential to the biochemistry of all cells and, in particular, for enzymatic activities associated with photosynthesis and with nitrogen fixation in the marine environment. Yet it is only required in very small quantities, as little as one part in 2000 compared to phosphorus
or one in 200,000 compared to carbon! 

As with macronutrients such as phosphate (\(PO^{2-}_{4}\)) and nitrate (\(NO^{-}_{3}\)), supply of \(Fe\) to the surface ocean occurs through up-welling and mixing of ocean waters from below. However, dissolved \(Fe\) has a short lifetime in the oxygenated seawater environment. \(Fe^{II}\), the most soluble state, is rapidly oxidized to \(Fe^{III}\), which is highly insoluble and tends to precipitate out and be removed (scavenged) by particulate matter settling through the water column. The result is that in up-welling water, the ratio of dissolved \(Fe\) to that of highly soluble \(PO^{2-}_{4}\) is lower than the ratio required by phytoplankton cells to grow and divide. Consequently, phytoplankton in surface waters cannot fully utilize the abundant up-welled
phosphate (or nitrate) unless more \(Fe\) is brought into the system.

Dust is important because mineral aerosols contain iron, primarily in the form of \(Fe\) oxides such as hematite and oxide-hydroxides such as goethite (found as coatings on other mineral grains). However, the present-day flux of aeolian Fe is not everywhere sufficient to correct the relative nutrient imbalance. For instance, dust fluxes to the Southern Ocean are among the lowest anywhere on Earth, and the aeolian \(Fe\) supply is too small to compensate for the depleted \(Fe\) relative to \(PO^{2-}_{4}\). Consequently, phytoplankton cannot fully utilize available macronutrients, and dissolved \(PO^{2-}_{4}\) exists year-round at the ocean surface. Similar reasoning applies to the presence of excess \(PO^{2-}_{4}\) in the Eastern Equatorial Pacific. Although dust supply to the North Pacific often appears moderately high in global model simulations, fertilization experiments in the Northwest Pacific suggest that this region is still iron-limited. Hence, the geography of complete nitrate utilization at the surface will be controlled to a first order by the mass distribution of dust deposition." \footnote{Ridgwell, A., The Global Dust Cycle, in Surface Ocean--Lower Atmospheres Processes, Eds. C. Le Quéré and E. S. Saltzman, AGU Geophysical Monograph Series, Volume 187, 350 pp.} \footnote{Also see: Jickells, T. D., Z. S. An, K. K. Andersen, A. R. Baker, G. Bergametti, N. Brooks, Cao J. J., P. W. Boyd, R. A. Duce, K. A. Hunter, H. Kawahata, N. Kubilay, J. laRoche, P. S. Liss, N. Mahowald, J. M. Prospero, A. J. Ridgwell, I. Tegen, and R. Torres, Global Iron Connections Between Desert Dust, Ocean Biogeochemistry and Climate, Science, 308, p. 67 (2005).}

%------------------------------------------------
\vspace{1mm} \noindent\rule{4cm}{0.1mm} \vspace{2mm}
%------------------------------------------------

%------------------------------------------------
\newpage
%------------------------------------------------

\noindent All experiments in this section start from the same \textit{re-start} and are based on the same \textit{user-config}\footnote{Remember to copy and rename the file \textsf{\footnotesize LAB.6.4.EXAMPLE} in order to create new and unique experiments each time.}:
\vspace{-2mm}\small\begin{verbatim}
./runcookie.sh cookie.CB.p_worjh2.BASESFe LABS LAB.6.4.EXAMPLE
 100 cookie.CB.p_worjh2.BASESFe.FeMIP.SPIN
\end{verbatim}\normalsize\vspace{-1mm}
This is effectively the iron cycle configuration used and evaluated in \textit{Tagliabue et al.} [2016]\footnote{Tagliabue, A., O. Aumont, R. DeAth, J.P. Dunne, S. Dutkiewicz, E. Galbraith, K. Misumi, J.K. Moore, A. Ridgwell, E. Sherman, C. Stock, M. Vichi, C. Völker, and A. Yool, How well do global ocean biogeochemistry models simulate dissolved iron distributions?, GBC DOI: 10.1002/2015GB005289 (2016).}.

\vspace{1mm}
The \textit{forcing} used in both the experiment includes a prescribed dust flux to the ocean surface (the \textsf{\footnotesize Mahowald} part of the directory name string). This is necessary because the model configuration you are using includes a co-limitation of biological productivity by iron (\(Fe\)) in addition to phosphate (\(PO_{4}\)). (The files associated with the dust forcing are: \textsf{\footnotesize biogem\_force\_flux\_sed\_det\_sig.dat} and \textsf{\footnotesize biogem\_force\_flux\_sed\_det\_SUR.dat} but you do not need to edit these files.) \textbf{BUT} -- note that no prescribed value of atmospheric \(pCO_{2}\) is given in the \textit{forcing} for this \textit{user-config} (although it was for generating the \textit{re-start}) -- this is so that any change you make to the marine iron cycle and hence to the biological pump in the ocean can be seen as an impact on atmospheric \(pCO_{2}\). This also helps make for a better control experiment, because if the \textit{re-start} was somehow mis-matched with the \textit{user-config}  or you accidently changed something you did not mean to (or notice), you could expect to immediately see atmospheric \(pCO_{2}\) starting to drift. (Also note that the experiments need not be run for 100 years ... you may want longer or shorter, depending on how you are perturbing the model and what exactly you are 'looking for'.)

%------------------------------------------------
\vspace{1mm} \noindent\rule{4cm}{0.1mm} \vspace{2mm}
%------------------------------------------------

\noindent What to change? (aka, what experiment/playing around to do?)

\begin{enumerate}[noitemsep]

\vspace{1mm}
\item Well ... firstly, you might check that biological productivity is in fact limited anywhere. You can try 2 different things:
\begin{enumerate}[noitemsep]
\vspace{1mm}
\item Reduce the half-saturation constant for \(Fe\) limitation to zero (so phytoplankton growth is never limited by iron, at least, up to the point where it completely runs out):
\vspace{-1mm}\small\begin{verbatim}
# [Fe] M-M half-sat value (mol kg-1)
bg_par_bio_c0_Fe=0.10E-09
\end{verbatim}\normalsize\vspace{-1mm}
and set this parameter to zero (or something very very small).
\vspace{1mm}
\item Flood the ocean with so much iron that it simply cannot be limiting anywhere.
\vspace{1mm}
\\The dust field itself is sort of difficult to do anything with (in terms of manipulations), so instead, you can change the assumed solubility of iron in dust -- i.e. what fraction of iron delivered to the surface ocean in dust is assumed to dissolve and hence become bio-available. The parameter controlling solubility\footnote{Note that this is a fractional solubility, not a \% solubility, so a value of \(0.001\) equates to a \(0.1\:wt\%\) solubility.} is:
\vspace{-1mm}\small\begin{verbatim}
# aeolian Fe solubility
bg_par_det_Fe_sol=0.00291468
\end{verbatim}\normalsize\vspace{-1mm}
And change to ... ? Maybe try an order of magnitude (or more) higher value.
\end{enumerate}
\vspace{1mm}
(The second modification is likely to have more impact and come closer to answering the question: what would export production and atmospheric \(pCO_{2}\) look like if there was no iron limitation on biological productivity? In the first option, you might still completely run out of iron somewhere in the ocean and hence limit biological productivity.)

%------------------------------------------------
\newpage
%------------------------------------------------

\vspace{1mm}
\item A second question/investigation (also utilizing a copy of \textit{user-config} \textsf{\footnotesize LAB.6.4.EXAMPLE}) might be to explore the internal cycling of iron and hence the importance of dissolved iron supplied 'from below' via ocean up-welling and mixing, rather 'from above' and directly from dust.

\vspace{1mm}
The dissolved \(Fe\) content of the ocean interior is set by: ocean transport (iron from elsewhere), iron released through the remineralization of organic matter, and iron scavenged from solution and removed onto sinking particles. of these, you have direct control over \(Fe\) scavenging, via:
\vspace{-1mm}\small\begin{verbatim}
# modifier of the scavenging rate of dissolved Fe
bg_par_scav_Fe_sf_POC=1.338130
\end{verbatim}\normalsize\vspace{-1mm}
that scales the rate at which \(Fe\) is removed from solution (this rate also depends on the sinking flux of particulate organic matter as well as the concentration of 'free' iron (not bound to organic ligands and hence assumed protected from scavenging).

\vspace{1mm}
Increasing this value will increase the loss rate of \(Fe\) from the ocean interior (and surface), while decreasing it will reduce the rate of loss. Consider an order of magnitude\footnote{It need not be any more than this. In fact, you might try a slightly smaller change, e.g. factor-\(5\).} change in value (in either direction) to create meaningful changes to the marine iron cycle.

\end{enumerate}

%------------------------------------------------
\vspace{1mm} \noindent\rule{4cm}{0.1mm} \vspace{2mm}
%------------------------------------------------

\noindent What to look for? Obviously, start with \(pCO_{2}\) (\textit{time-series}) and \(POC\) export (which can be viewed as both \textit{time-series} and spatially in the 2D \textbf{netCDF} output). (All of these are also included in the summary output files.) Also 2D distributions of \(PO^{2-}_{4}\) and \(Fe\) are highly relevant and useful. For more involved outputs, also in the  2D \textbf{netCDF} output you can find the pattern of iron solubility (\textsf{\footnotesize misc\_sur\_Fe\_sol}) and the flux of dissolved iron to the ocean surface (\textsf{\footnotesize misc\_sur\_fFe\_mol}). 

\vspace{1mm}
Also, in the  2D \textbf{netCDF} output -- \textsf{\footnotesize misc\_sur\_PO4Felimbalance} -- is a simple attempt to illustrate where \(PO^{2-}_{4}\) (positive values) rather than \(Fe\) limitation of biological productivity occurs. For example, high positive values (on scale up to \(1.0)\) can be found in the Equatorial Atlantic and Indian Oceans, suggesting plenty or dissolved iron but no phosphate. The Southern Ocean is more iron than phosphate limited, while the Pacific gyres (most negative values) are dominantly iron limited.

%------------------------------------------------
\vspace{1mm} \noindent\rule{4cm}{0.1mm} \vspace{2mm}
%------------------------------------------------

%------------------------------------------------
\newpage
%------------------------------------------------

\section{Further ideas}

%------------------------------------------------
%------------------------------------------------

\subsection{Further analysis ideas}

\noindent In addition to the primary environmental properties of atmospheric \(pCO_{2}\) (and climate if you have the feedback enabled), biological export production (and hence the patterns and magnitude of of food sources for marine ecosystems), and dissolved oxygen, you might also look at how the patterns of carbon isotopes (\(\delta^{13}C\)) (particularly of \(DIC\)) change in the ocean. 

How did the different changes in the biological pump that you tested, alter (if at all) the patterns of \(\delta^{13}C\) in the ocean? Can you distinguish between the different biological pump changes, based on \(\delta^{13}C\)? (Carbonate and organic carbon \(\delta^{13}C\) is a key paleoceanographic proxy and one people would ideally like to use in order to reconstruct changes in the past such as in the biological pump.)

%------------------------------------------------
%------------------------------------------------

\subsection{Further experiment ideas}

\noindent By default, the update and export in organic matter, of carbon, occurs in a fixed 'Redfield' ratio, with \(PO^{2-}_{4}\). The classical value is \(106\) (i.e., for every mole of \(PO^{2-}_{4}\) taken up from the ocean, \(106\) moles of \(CO_{2}\) are removed (and fixed into organic matter). The parameter determining this is:
\vspace{-2mm}\small\begin{verbatim}
bg_par_bio_red_POP_POC=106.0
\end{verbatim}\normalsize\vspace{-2mm}
To change the default value (106.0), simply add a new line at the end of the \textit{user-config} file specifying the value you want. A larger number means that \(PO_{4}\) is being utilized more efficiently and more organic matter is being produced for the same nutrient consumption.

You should see impacts on atmospheric \(pCO_{2}\) and the oxygenation of the ocean interior form changing this.

%------------------------------------------------
\vspace{1mm} \noindent\rule{4cm}{0.5pt} \vspace{2mm}
%------------------------------------------------

\noindent To test the effect of there being more (or less) \(PO^{2-}_{4}\) in the ocean in the first place, it is possible to increase the inventory of the ocean as a whole follwing a re-start\textit{, }by:
\vspace{-2mm}\small\begin{verbatim}
bg_ocn_dinit_8=1.0E-6
\end{verbatim}\normalsize\vspace{-2mm}
which will add \(1\:\mu mol\:kg^{-1}\) of \(PO^{2-}_{4}\) uniformly to the ocean. (A larger/smaller number will obviously increase the glacial nutrient inventory by more/less. A negative number will remove \(PO^{2-}_{4}\).)

%------------------------------------------------
\vspace{1mm} \noindent\rule{4cm}{0.5pt} \vspace{2mm}
%------------------------------------------------

\noindent You might also play with the 'half saturtaion constant' for nutrient uptake (see \textit{Ridgwell et al.} [2007]):
\vspace{-2mm}\small\begin{verbatim}
#[PO4] M-M half-sat value (mol kg-1)
bg_par_bio_c0_PO4=2.1989611E-07
\end{verbatim}\normalsize\vspace{-1mm}
As set, this specifies that at a \(PO^{2-}_{4}\) concentration of about \(0.2 \mu mol \:kg^{-1} \), growth (net export) is half the maximum possible.

\vspace{1mm}
You might, for instance, try setting the value to zero, which assumes maximum growth can continue right up nutrient completely running out. This would be similar to assuming all the primary producers in the ocean had extremely small cell sizes (and hence low half saturation values) and were adapted to oligotrophic conditions.

Or, test an ocean dominated by assumed very large cell sizes and phytoplankton that struggle under anything other than fully eutrophic (high nutrient) conditions. e.g. you might try values of \(1.0 \mu mol \:kg^{-1}\) (\texttt{1.0E-06}) or even \(2.0 \mu mol \:kg^{-1}\) (\texttt{2.0E-06}) and see what happens, particularly to the pattern of surface nutrient concentrations and export.

%------------------------------------------------
\vspace{1mm} \noindent\rule{4cm}{0.5pt} \vspace{2mm}
%------------------------------------------------

%------------------------------------------------
\newpage
%------------------------------------------------

\noindent There is one more knob controlling organic matter export and cycling in the ocean that you could tweak and explore the effect of the assumption regarding how much organic matter produced is partitioned into particulate form that sinks, vs. dissolved, controlled by:
\vspace{-2mm}\small\begin{verbatim}
bg_par_bio_red_DOMfrac=0.66
\end{verbatim}\normalsize\vspace{-2mm}
which specifies 66\% of organic matter production is diverted into dissolved form. See \textit{Ridgwell and Arndt} [2014]\footnote{Ridgwell, A., and S. Arndt, Why Dissolved Organics Matter: DOC in Ancient Oceans and Past Climate Change, in: Biogeochemistry of Marine Dissolved Organic Matter Eds. Hansell, D. A., and C. A. Carlson, Elsevier (2014).}. And/or, you might adjust its mean lifetime in the ocean, which by default is  \(0.5 yr\):
\vspace{-2mm}\small\begin{verbatim}
bg_par_bio_remin_DOMlifetime=0.5
\end{verbatim}\normalsize\vspace{0mm}

%------------------------------------------------
%------------------------------------------------

\subsection{Other thoughts and suggestions}

\begin{itemize}

\vspace{1mm}
\item If you want to combine forcings, you need to first update the file: \textsf{\footnotesize configure\_forcings\_ocn.dat} – this specifies which ocean flux forcing will be used – simply copy the relevant line from the equivalent file of the forcing to be added. You will also need to copy in the relevant ‘\texttt{\_sig.dat}’ and ‘\texttt{\_SUR.dat}’ files. Remember that in the \textit{user-config} file, you will need to set the relevant flux scaling parameter for each different flux in the forcing.

\vspace{1mm}
\item By default, the \(CO_{2}\)-climate feedback is ‘on’:
\vspace{-2pt}\small\begin{verbatim}
# set climate feedback
ea_36=y
\end{verbatim}\normalsize\vspace{-2pt}
Should you want to assess the impacts of geoengineering independently of changes in climate -- the option is there. (Note that under some of the high end \(CO_{2}\) emissions scenarios, there may be a degree of collapse of the AMOC that will presumably affect the patterns of ocean acidification and oxygenation etc.)

\vspace{1mm}
\item If you are having doubts that your experiment is actually ‘doing’ anything (different from the control) – remember to create anomaly maps (plots) to look for specific changes in e.g. saturation state, pH, or the water column inventory of anthropogenic \(CO_{2}\). Even before this – plot anomalies of the flux you think you have applied, looking specifically at the region you think you have applied it to. For this, \textbf{cookie} saves the 3D distributions of dissolved Fe and \(PO_{4}\). See Figures below.

\vspace{1mm}
\item Always be aware of the caveats regarding this specific model (and models in general) – how much does it different form the ‘real world’ for the modern ocean, particularly in terms of patterns of carbonate saturation? Does it even simulate anthropogenic \(CO_{2}\) uptake adequately in the first place?

\end{itemize}

%------------------------------------------------
%------------------------------------------------

%------------------------------------------------
\newpage 
%------------------------------------------------

\subsection{Biological pump background information}

A few notes and background information on controls on the parameterized controls on the biological pump in \textbf{cookie}.

%------------------------------------------------
\vspace{1mm} \noindent\rule{4cm}{0.5pt} \vspace{2mm}
%------------------------------------------------

\subsubsection*{Remineralization depth}

\vspace{2mm}
In the model configuration that you have been using, the degradation of particulate organic matter sinking in the water column proceeds according to a fixed profile of flux with depth (there is no e.g. temperature control on the rate of bacterial degradation of sinking organic matter) with \(CO_{2}\) and \(PO_{4}\) released back to the seawater as the particulate flux decreases. The parameter that controls the (\textit{e}-folding) depth scale of particulate organic matter is:
\vspace{-1mm}\small\begin{verbatim}
bg_par_bio_remin_POC_eL1=589.9451
\end{verbatim}\normalsize\vspace{-1mm}
Either edit this value (found under the heading: \texttt{\# --- REMINERALIZATION ---}) or add a new line at the end of the \textit{user-config} file specifying the value you want. Units are \(m\).

\vspace{1mm}
Read \textit{Ridgwell et al.} [2007] for additional discussion of this parameter. See Figure 2-4 in \textit{Ridgwell} [2001] (http://www.seao2.org/pubs/ridgwell\_thesis.pdf) for an illustration of how the flux of particulate organic matter decreases with depth in the ocean, plus references therein.

There is also an associated parameter: \texttt{bg\_par\_bio\_remin\_POC\_frac2}, which sets a fraction of organic matter that is assumed to settle through the water column completely un-altered (currently assigned a value of 0.045 == 4.5\%), but this is arguably less useful to change than the remineralization length-scale of the more labile fraction (the other 95.5\% of particulate organic carbon exported from the ocean surface).

%------------------------------------------------
\vspace{1mm} \noindent\rule{4cm}{0.5pt} \vspace{2mm}
%------------------------------------------------

\subsubsection*{CaCO3:POC rain ratio}

\vspace{2mm}
Kicked off by a classic 1994 \textit{Nature} paper by \textit{Archer and Maier-Reimer} (see: \textit{Kohfeld and Ridgwell} [2009]), one potential means of changing atmospheric \(CO_{2}\) naturally at the last glacial involves changes in the export ratio between \(CaCO_{3}\) (shells) and \(POC\) (particulate organic matter). Such a change in ratio could come about through a variety of ways (e.g., via the 'silica leakage hypothesis' (see: \textit{Kohfeld and Ridgwell} [2009]) and also through the direct effect of \(Fe\) on diatom physiology (see \textit{Watson et al.} [2000] in \textit{Nature} and also Supplemental Information). There are also ideas about an opposite ocean acidification effect, whereby the less acidic glacial (compared to modern) ocean led to increased calcification and \(CaCO_{3}\) export. Note that this response (higher saturation == greater rate of calcification) is encoded into your model configuration – see \textit{Ridgwell et al.} [2007b].

\vspace{1mm}
In \textbf{cookie}, the \(CaCO_{3}:POC\) rain ratio is controlled (technically: scaled) by the parameter:
\vspace{-1mm}\small\begin{verbatim}
bg_par_bio_red_POC_CaCO3=0.0485
\end{verbatim}\normalsize\vspace{-1mm}

The pattern of \(CaCO_{3}:POC\) rain ratio is not uniform across the ocean (why? (see: \textit{Ridgwell et al.} [2007, 2009]), and its pattern can be viewed in the (2D \textbf{BIOGEM}) netCDF variable: \textsf{\footnotesize misc\_sur\_rCaCO3toPOC}.

%------------------------------------------------
\newpage 
%------------------------------------------------

\subsection{Advanced experiment ideas ...}

In today's ocean, iron limitation of biological productivity is only regionally important (compared to e.g. in the first exercise you were changing the dissolved iron supply from dust \uline{everywhere}). So one might legitimately ask whether specific region are iron limited, and how important this is (or not) to atmospheric \(pCO_{2}\). However, changing the dust field is not trivial, or rather it would be extremely time-consuming to e.g. edit the numbers to double all the dust fluxes in a specific region (e.g. the Southern Ocean).

\vspace{1mm}
As an alternative, we could re-configure the model to read in a field specifying the spatial pattern of iron solubility, set the values of iron solubility to a uniform value (\(0.1\%\)), and then simply edit the values in the file to create whatever pattern of modified iron input you want.

%
To do this, you need to add the following lines to the *end* of the \textit{user-config} file\footnote{Ideally, we would replace or delete a few parameter settings in the \textit{user-config} file in order to avoid confusion or mistakes, but this way is much simpler.}:

\vspace{-1mm}\small\begin{verbatim}
# Replace internal dust Fe solubility field?
bg_ctrl_force_det_Fe_sol=.true.
# Filename for dust Fe solubility field
bg_par_det_Fe_sol_file='worjh2.det_Fe_sol.MahowaldUNIFORMsol.dat'
\end{verbatim}\normalsize\vspace{-1mm}
which tells \textbf{cookie} to use a prescribed field of solubility values rather than calculating it, and then directs \textbf{cookie} to the file: \textsf{\footnotesize worjh2.det\_Fe\_sol.MahowaldUNIFORMsol.dat} containing the spatial pattern of solubility values.

\vspace{1mm}
This file can be found in the directory: \textsf{\footnotesize genie-biogem/data/input} and can be edited to change (increase or decrease) the solubility of iron in dust in specific regions (hence changing the dissolved iron flux). 

It is important to note that the values in this file are \% father than fractional solubility. By default, the file is populated with value \(0.1\) everywhere (at all ocean grid points) -- \(0.1\%\), or a fractional solubility of \(0.001\) (as per was the value of the parameter \texttt{\small bg\_par\_det\_Fe\_sol} before).

\vspace{1mm}
Lastly, we need to modify the iron scavenging rate:
\vspace{-1mm}\small\begin{verbatim}
# modifier of the scavenging rate of dissolved Fe
bg_par_scav_Fe_sf_POC=0.1
\end{verbatim}\normalsize\vspace{-1mm}
to match the change in iron solubility.

\vspace{1mm}
A little care will be needed in running experiments. The change in pattern of iron solubility and  scavenging rate means that the iron cycle will be slightly different as compared to the \textit{re-start} and when you start running it, the distribution of dissolved iron and hence iron availability to biology will start to change, in turn changing the biological pump and atmospheric \(pCO_{2}\). You could either run a control and subtract the drift in whatever parameters you are interested in friom the real experiment, or better, spin the adjusted configuration up. For the latter, you'd take the \textit{re-start} as before, and maybe run e.g. 1000 years under the new iron parameter values and see whether the system is re-equilibrating. If this looks 'good', simply use this (e.g. 1000 year) experiment as your new \textit{re-start}.

\vspace{1mm}
Increasing or decreasing the iron input to different regions of the ocean surface is simply then a matter of editing the values in \textsf{\footnotesize worjh2.det\_Fe\_sol.MahowaldUNIFORMsol.dat} in some pattern (hopefully informed by some hypothesis that you have formulated about iron limitation and the biological pump).

%----------------------------------------------------------------------------------------
%----------------------------------------------------------------------------------------